\documentclass[11pt,a4paper]{article}

%--- Encodage & langue ---
\usepackage[T1]{fontenc}
\usepackage[utf8]{inputenc}
\usepackage[french,english]{babel}
\usepackage{lmodern}
\usepackage{microtype}

%--- Mise en page ---
\usepackage[a4paper,margin=2cm]{geometry}
\usepackage{setspace}
\setstretch{1.08}

%--- Couleurs & liens ---
\usepackage{xcolor}
\definecolor{main}{HTML}{0A6CF1}
\definecolor{accent}{HTML}{FF6B6B}
\definecolor{soft}{gray}{0.15}
\usepackage[
  colorlinks=true,
  linkcolor=main,
  urlcolor=main,
  citecolor=main
]{hyperref}

%--- Listes propres ---
\usepackage{enumitem}
\setlist[itemize]{topsep=2pt,itemsep=2pt,parsep=0pt}
\setlist[enumerate]{topsep=2pt,itemsep=2pt,parsep=0pt}

%--- Boîtes et encadrés ---
\usepackage[most]{tcolorbox}
\tcbset{sharp corners, boxrule=0.4pt, colframe=main!70!black, colback=main!3}

% Styles de boîtes
\newtcolorbox{RuleBox}{colframe=main!80!black, colback=main!3, title=\textbf{Règle}}
\newtcolorbox{TipBox}{colframe=green!50!black, colback=green!5, title=\textbf{Conseil}}
\newtcolorbox{WarnBox}{colframe=accent!80!black, colback=accent!5, title=\textbf{À éviter}}
\newtcolorbox{ExampleBox}{colframe=soft, colback=black!3, title=\textbf{Exemples}}

%--- Tableaux ---
\usepackage{booktabs}
\usepackage{bookmark}
\usepackage{array}
\newcolumntype{L}[1]{>{\raggedright\arraybackslash}p{#1}}

\newcounter{rownum}
\newcommand{\resetrownum}{\setcounter{rownum}{0}}
\newcommand{\rownumber}{\stepcounter{rownum}\arabic{rownum}}

%--- Commandes utiles ---
\newcommand{\Rule}[1]{\begin{RuleBox}#1\end{RuleBox}}
\newcommand{\Tip}[1]{\begin{TipBox}#1\end{TipBox}}
\newcommand{\Warn}[1]{\begin{WarnBox}#1\end{WarnBox}}
\newcommand{\Exemples}[1]{\begin{ExampleBox}#1\end{ExampleBox}}

%--- Titre ---
\title{\vspace{-1.5em}\textbf{Préparation TOEIC}\\
\large Grammaire \;|\; Vocabulaire \;|\; Conseils pratiques \;|\; Tips Toeic}
\author{%
Mohamed Lemine Mohamed Ahmed \\
\small \href{mailto:prenom.nom@email.com}{mhmdahmdnbyl18@gmail.com}
}
\date{\small \today}

\begin{document}
\selectlanguage{french}
\maketitle
\vspace{-1em}
\hrule
\vspace{0.8em}

\tableofcontents
\newpage


%========================
\section{Grammaire essentielle}
%========================

\subsection{Tenses}

\subsubsection{Present simple}


\subsubsection*{a. Formulation}

\paragraph{Affirmative form}
\begin{itemize}
  \item \textbf{I / You / We / They} + \textit{base verb}  
  \textit{Example:} I work every day.
  \item \textbf{He / She / It} + \textit{verb + s / es}  
  \textit{Example:} She works in a bank.
\end{itemize}

\paragraph{Negative form}
\begin{itemize}
  \item \textbf{I / You / We / They} + \textit{do not (don’t) + base verb}  
  \textit{Example:} They don’t like coffee.
  \item \textbf{He / She / It} + \textit{does not (doesn’t) + base verb}  
  \textit{Example:} He doesn’t play football.
\end{itemize}

\paragraph{Interrogative form}
\begin{itemize}
  \item \textbf{Do} + \textit{I / you / we / they} + \textit{base verb} ?  
  \textit{Example:} Do you speak English?
  \item \textbf{Does} + \textit{he / she / it} + \textit{base verb} ?  
  \textit{Example:} Does she study French?
\end{itemize}

\subsubsection*{b. The verbs \textit{to be} and \textit{to have}}

\paragraph{To be}
\begin{itemize}
  \item \textbf{Affirmative:} I am / You are / He-She-It is / We are / They are  
  \textit{Example:} She is a teacher.
  \item \textbf{Negative:} I am not / You aren’t / He isn’t / etc.  
  \textit{Example:} They aren’t tired.
  \item \textbf{Interrogative:} Am I...? / Are you...? / Is he...?  
  \textit{Example:} Are you ready?
\end{itemize}

\paragraph{To have}
\begin{itemize}
  \item \textbf{Affirmative:} I/You/We/They have – He/She/It has  
  \textit{Example:} He has a car.
  \item \textbf{Negative:} I/You/We/They don’t have – He/She/It doesn’t have  
  \textit{Example:} She doesn’t have time.
  \item \textbf{Interrogative:} Do you have...? / Does he have...?  
  \textit{Example:} Do they have children?
\end{itemize}

\subsection*{c. Uses of the Present Simple}

\begin{itemize}
  \item \textbf{General facts and situations:}  
  \textit{Example:} The sun rises in the east.
  \item \textbf{Regular actions and routines:}  
  \textit{Example:} I go to the gym on Mondays.
  \item \textbf{With adverbs of frequency:} always, usually, often, sometimes, rarely, never.  
  \textit{Example:} She always drinks tea in the morning.
  \item \textbf{Universal or scientific truths:}  
  \textit{Example:} Water boils at 100°C.
  \item \textbf{Timetables and scheduled events:}  
  \textit{Example:} The train leaves at 9 a.m.
\end{itemize}


\subsubsection{Present Continuous (Progressive)}


\subsubsection*{a. Formulation}

\paragraph{Structure}
\[
\text{Subject} + \text{to be (am / is / are)} + \text{verb} + \text{-ing}
\]

\textbf{Examples:}
\begin{itemize}
  \item I am studying English.
  \item She is watching TV.
  \item They are working on a new project.
\end{itemize}

\paragraph{Negative form}
\[
\text{Subject} + \text{am / is / are not} + \text{verb} + \text{-ing}
\]
\textit{Examples:}  
He isn’t listening. / We aren’t playing football.

\paragraph{Interrogative form}
\[
\text{Am / Is / Are} + \text{subject} + \text{verb} + \text{-ing} \, ?
\]
\textit{Examples:}  
Are you coming? / Is she reading now?

\subsubsection*{b. Uses of the Present Continuous}

\begin{itemize}
  \item \textbf{Actions happening now:}  
  \textit{Example:} She is talking on the phone right now.
  \item \textbf{Ongoing projects or activities:}  
  \textit{Example:} We are building a new website this month.
  \item \textbf{Temporary situations:}  
  \textit{Example:} I am living with my parents for a few weeks.
  \item \textbf{Changing or developing situations (trends, evolution):}  
  \textit{Example:} The climate is getting warmer.
  \item \textbf{Future plans or arrangements (near future):}  
  \textit{Example:} I’m meeting her soon.
\end{itemize}

\subsubsection*{c. TOEIC Tips}

\subsubsection*{- With time adverbs}

The Present Continuous is almost always used with time adverbs.

\textbf{Common time adverbs:}  
currently, at the moment, this year, this week, today, still, these days, now, meanwhile, right now.

\textbf{Examples:}
\begin{itemize}
  \item She is studying for her exams at the moment.
  \item We are planning a trip to Japan this year.
  \item He is still working on the project right now.
\end{itemize}

\subsubsection*{- With \textit{always} to express criticism}

Normally, \textit{always} is used with the Present Simple.  
However, when expressing irritation, annoyance, or a negative emotion, we use the Present Continuous.

\textbf{Examples:}
\begin{itemize}
  \item He always loses his keys. \hfill (Present Simple — it’s habitual)
  \item He is always losing his keys! \hfill (Present Continuous — it’s annoying!)
\end{itemize}


\subsubsection{Present Perfect Simple}



\subsection{Nouns : }

Nouns


\subsection{Tips for TOEIC}



\begin{table}[h]
\centering
\resetrownum
\begin{tabular}{@{}p{0.5cm} p{4cm} p{6.5cm} p{6.5cm}@{}}
\toprule
\text{Num} & \textbf{Word / Structure} & \textbf{Definition } & \textbf{Example} \\
\midrule
\rownumber & \textit{under which} & on utilise \textbf{under which} pour parler d'une règle ou loi & 
The policy \textbf{under which} refunds are processed has changed. \\
\midrule
\rownumber & \textit{prefer + V-ing + to + V-ing} & Compare two activities / preferences & 
I prefer \textbf{reading} reports to \textbf{attending} long meetings. \\
\midrule
\rownumber & \textit{has/have been + past participle} & Present perfect passive: past action with a present result & 
The contract \textbf{has been signed}. \\
\midrule
\rownumber & \textit{delighted with} & Pleased/very happy about something &
The client was \textbf{delighted with} the results. \\
\midrule
\rownumber & \textit{jewellery} (uncountable) & Uncountable noun; no plural *jewelleries*; verb takes singular &
Her \textbf{jewellery is} expensive. \\
\midrule
\rownumber & \textit{once} + present simple & Use present simple after “once” even for future reference &
\textbf{Once he arrives}, we will start. \\
\midrule
\rownumber & \textit{could} (possibility) & Express possibility/conditional suggestion &
If we had more time, we \textbf{could} expand the study. \\
\midrule
\rownumber & \textit{object to} + N / V-ing & “Object” always with \textit{to}; if a verb follows, use -ing &
They objected \textbf{to} the proposal / \textbf{to working} late. \\
\midrule
\rownumber & \textit{be keen on} + N/V-ing; \textit{keen to} + V & Enthusiasm/interest (\textit{keen on}); willingness/intent (\textit{keen to}) &
She is \textbf{keen on learning} Python; she is \textbf{keen to learn} Python. \\
\bottomrule
\end{tabular}
\end{table}



%========================
\section{Vocabulaire \& collocations}
%========================
\begin{center}
\begin{tabular}{@{}L{4.5cm} L{10cm}@{}}
\toprule
\textbf{Expression} & \textbf{Exemple TOEIC} \\
\midrule
meet a deadline & We must \textbf{meet the deadline} by Friday. \\
make a decision & Management will \textbf{make a decision} tomorrow. \\
apply for a position & She \textbf{applied for} the marketing position. \\
be responsible for & He is \textbf{responsible for} quality control. \\
in accordance with & The policy is \textbf{in accordance with} regulations. \\
\bottomrule
\end{tabular}
\end{center}

%========================
\section{Contractions \& connecteurs}
%========================

\subsection*{Contractions fréquentes (écoute TOEIC)}
\begin{center}
\begin{tabular}{@{}L{5cm} L{9.5cm}@{}}
\toprule
\textbf{Forme} & \textbf{Exemple} \\
\midrule
I'm, you're, he's, we're, they're & \textit{They're ready for the call.} \\
don't, doesn't, didn't & \textit{She doesn't agree.} \\
I'll, you'll, we'll, they'll & \textit{I'll send the report.} \\
can't, won't, shouldn't, couldn't & \textit{We can't attend today.} \\
\bottomrule
\end{tabular}
\end{center}

\subsection*{Connecteurs logiques (écrit/oral)}
\begin{itemize}
  \item \textbf{Therefore, consequently}: conséquence \;—\; \textit{It was delayed; therefore, we rescheduled.}
  \item \textbf{However, nevertheless}: opposition \;—\; \textit{Expensive; however, effective.}
  \item \textbf{Moreover, in addition}: addition \;—\; \textit{Moreover, we reduced costs.}
\end{itemize}

%========================
\section{Conseils d’examen \& stratégies}
%========================
\Tip{
\textbf{Timing}: ne t’attarde pas — 1 question difficile = passe et reviens plus tard. \\
\textbf{Listening}: anticipe le contexte (lieu, rôle, objectif) avant l’audio. \\
\textbf{Reading}: lis d’abord les questions, puis le texte (gain de temps). \\
\textbf{Grammaire}: repère les indices (prépositions fixes, structure des temps). \\
\textbf{Vocabulaire}: privilégie collocations et expressions métiers.
}

\Warn{
Évite de sur-analyser. Les distracteurs typiques au TOEIC : synonymes proches, négations cachées, dates/chiffres piégés.
}

%========================
\section{Mini-fiches \& pièges classiques}
%========================
\begin{itemize}
  \item \textbf{After / before / once / when} + présent simple pour futur proche.
  \item \textbf{Depend on}, \textbf{insist on}, \textbf{apologize for}, \textbf{object to}.
  \item \textbf{Few} (peu, négatif) vs \textbf{a few} (quelques, positif).
  \item \textbf{Since} (point de départ) vs \textbf{for} (durée).
\end{itemize}

% %========================
% \section{Checklist de révision rapide}
% %========================
% \begin{itemize}
%   \item[{\large $\square$}] 20 collocations métier révisées
%   \item[{\large $\square$}] 15 connecteurs logiques maîtrisés
%   \item[{\large $\square$}] 10 pièges de grammaire cochés
%   \item[{\large $\square$}] 2 tests blancs chronométrés
% \end{itemize}


\vfill
\begin{center}
\small \textit{Dernière mise à jour : \today \;—\; Ce document est un mémo de révision TOEIC.}
\end{center}

\end{document}
