\documentclass[11pt,a4paper]{report}

%--- Encodage & langue ---
\usepackage[T1]{fontenc}
% \usepackage{fontspec}   % pour gérer les polices Unicode

\usepackage[utf8]{inputenc}
\usepackage[french,english]{babel}
\usepackage{lmodern}
\usepackage{microtype}

%--- Mise en page ---
\usepackage[a4paper,margin=2cm]{geometry}
\usepackage{setspace}
\setstretch{1.08}

%--- Couleurs & liens ---
\usepackage{xcolor}
\definecolor{main}{HTML}{0A6CF1}
\definecolor{accent}{HTML}{FF6B6B}
\definecolor{soft}{gray}{0.15}
\usepackage[
  colorlinks=true,
  linkcolor=main,
  urlcolor=main,
  citecolor=main
]{hyperref}



%--- Listes propres ---
\usepackage{enumitem}
\setlist[itemize]{topsep=2pt,itemsep=2pt,parsep=0pt}
\setlist[enumerate]{topsep=2pt,itemsep=2pt,parsep=0pt}

%--- Boîtes et encadrés ---
\usepackage[most]{tcolorbox}
\tcbset{sharp corners, boxrule=0.4pt, colframe=main!70!black, colback=main!3}

% Styles de boîtes
\newtcolorbox{RuleBox}{colframe=main!80!black, colback=main!3, title=\textbf{Règle}}
\newtcolorbox{TipBox}{colframe=green!50!black, colback=green!5, title=\textbf{Conseil}}
\newtcolorbox{WarnBox}{colframe=accent!80!black, colback=accent!5, title=\textbf{À éviter}}
\newtcolorbox{ExampleBox}{colframe=soft, colback=black!3, title=\textbf{Exemples}}

%--- Tableaux ---
\usepackage{booktabs}
\usepackage{bookmark}
\usepackage{array}
\newcolumntype{L}[1]{>{\raggedright\arraybackslash}p{#1}}

\usepackage{longtable}

\usepackage{polyglossia} % pour la gestion des langues
\setmainlanguage{english} % langue principale
\setotherlanguage{arabic} % ajouter l'arabe
\newfontfamily\arabicfont[Script=Arabic]{Amiri} % ou une autre police arabe installée


\newcounter{rownum}
\newcommand{\resetrownum}{\setcounter{rownum}{0}}
\newcommand{\rownumber}{\stepcounter{rownum}\arabic{rownum}}

%--- Commandes utiles ---
\newcommand{\Rule}[1]{\begin{RuleBox}#1\end{RuleBox}}
\newcommand{\Tip}[1]{\begin{TipBox}#1\end{TipBox}}
\newcommand{\Warn}[1]{\begin{WarnBox}#1\end{WarnBox}}
\newcommand{\Exemples}[1]{\begin{ExampleBox}#1\end{ExampleBox}}

%--- Titre ---
\title{\vspace{-1.5em}\textbf{Préparation TOEIC}\\
\large Grammaire \;|\; Vocabulaire \;|\; Conseils pratiques \;|\; Tips Toeic}
\author{%
Mohamed Lemine Mohamed Ahmed \\
\small \href{mailto:prenom.nom@email.com}{mhmdahmdnbyl18@gmail.com}
}
\date{\small \today}

\begin{document}
% \selectlanguage{french}
\maketitle
\vspace{-1em}
\hrule
\vspace{0.8em}

\tableofcontents
\newpage


%========================
% Grammaire 
%========================


\begin{center}
    {\Huge \textbf{Grammaire essentielle}}\\[6pt]
    \hrule
\end{center}

\vspace{1em}


% ========= chapitre : Parts of Speech ================

\chapter{Parts of Speech}

\section{Nouns }


\subsection*{a. Definition}

Nouns are the names of people, places, and things.

\noindent
\textbf{Common Suffixes :}
\vspace{0.5cm}

\begin{tabular}{l l l}
\textbf{Suffix} & \textbf{Meaning} & \textbf{Example} \\
\hline
-ness & state, condition & happiness, kindness \\
-dom & state, domain & freedom, kingdom \\
-hood & state, group & adulthood, neighbourhood \\
-ship & relationship, state & friendship, internship \\
\end{tabular}

\subsection*{b. Singular \& Plural Rules}
\begin{tabular}{l l l}
\textbf{Rule} & \textbf{Singular} & \textbf{Plural} \\
\hline
Add -s & car & cars \\
Ends in s, ss, x, sh, ch $\rightarrow$ +es & watch & watches \\
Consonant + y $\rightarrow$ ies & baby & babies \\
Vowel + y $\rightarrow$ +s & boy & boys \\
Ends in o $\rightarrow$ +es & potato & potatoes \\
Ends in o $\rightarrow$ +s & radio & radios \\
Ends in f/fe $\rightarrow$ ves/s & shelf & shelves \\
Irregular & man & men \\
Irregular & child & children \\
Irregular & mouse & mice \\
\end{tabular}


\subsection*{c. Articles }

\textbf{Definition :}

Articles are small words placed before nouns to show whether the noun is general or specific. They help indicate if we are talking about any member of a group (a, an) or a particular one that is already known to the listener (the)

\vspace{0.5cm}


\begin{tabular}{l l l}
\textbf{Article} & \textbf{Use} & \textbf{Example} \\
\hline
a & Before singular countable nouns starting with a consonant & a dog, a sandwich \\
an & Before singular countable nouns starting with a vowel & an orange, an apple \\
the & Specific noun (singular or plural, countable or uncountable) & the book, the books, the water \\
\end{tabular}



\subsection*{d. Demonstratives}

\textbf{Definition :}

Demonstratives are words used to point to specific people, things, or places. They show whether something is near or far and whether it is singular or plural.
The main demonstratives in English are: this, that, these, those.

\vspace{0.5cm}

\begin{tabular}{l l l}
\textbf{Word} & \textbf{Use} & \textbf{Example} \\
\hline
this & Singular, near the speaker & this book \\
that & Singular, far from the speaker & that car \\
these & Plural, near the speaker & these apples \\
those & Plural, far from the speaker & those houses \\
\end{tabular}




\section{Countable \& Uncountable Nouns}

\subsection*{a. Countable nouns}
Countable nouns are nouns that can be \emph{counted} as individual units. They normally have both singular and plural forms, and can be used with numbers and the indefinite articles \textit{a}/\textit{an}.


\paragraph{Frequent TOEIC examples:}
\begin{itemize}
  \item \textbf{job / jobs}, \textbf{report / reports}, \textbf{meeting / meetings}
  \item \textbf{customer / customers}, \textbf{product / products}
  \item \textbf{employee / employees}, \textbf{task / tasks}, \textbf{problem / problems}
\end{itemize}

\subsection*{b. Uncountable nouns}
Uncountable nouns  are nouns that cannot be counted as separate individual units without a unit/measure word. They normally do not have a plural form and cannot directly take \textit{a}/\textit{an}.


\paragraph{Frequent TOEIC examples:}
\begin{itemize}
  \item \textbf{information}, \textbf{advice}, \textbf{equipment} , \textbf{news}
  \item \textbf{money}, \textbf{luggage}, \textbf{research}
  \item \textbf{traffic}, \textbf{software}, \textbf{work}
  \item \textbf{furniture}
\end{itemize}

\subsection*{c. Quantifiers}

\subsubsection*{Definition :}
Quantifiers are words that express quantity. Some quantifiers are used with \emph{countable} nouns, some with \emph{uncountable} nouns, and some with both. 



\subsubsection*{Table of quantifiers :}

\vspace{0.5em}
\begin{tabular}{@{} l p{5cm} p{7.5cm} @{}}
\toprule
\textbf{Quantifier} & \textbf{Use} & \textbf{Example } \\
\midrule
many      & Countable & \textit{There are many customers waiting at reception.} \\
few       & Countable & \textit{Few applicants met the job requirements.} \\
a few     & Countable & \textit{I need a few minutes to finish the report.} \\
\hline
much      & Uncountable & \textit{We don’t have much time before the deadline.} \\
little    & Uncountable & \textit{There is little evidence to support the claim.} \\
a little  & Uncountable & \textit{We have a little information about the client.} \\
\hline
a lot of  & Countable \& Uncountable &  \textit{A lot of traffic delayed the shipment.} \\
some      & Countable \& Uncountable  &  \textit{Would you like some coffee?} \\
any       & Countable \& Uncountable  &  \textit{We don’t have any equipment available.} \\
\bottomrule
\end{tabular}



% ============ chapitre Tenses : =================
\chapter{Tenses }


\section{Present simple}



\subsection*{a. Formulation}


\[
\left\{
\begin{array}{ll}
\textbf{Affirmative:} & \text{Sujet + base verbale (+ s à la 3\textsuperscript{e} personne)} \\[6pt]
\textbf{Negative:} & \text{Sujet + do/does + not + base verbale} \\[6pt]
\textbf{Interrogative:} & \text{Do/Does + sujet + base verbale ?}
\end{array}
\right.
\]


\vspace{0.5em}
\textbf{Example :  to work }
\begin{center}
\renewcommand{\arraystretch}{1.3}
\begin{tabular}{l|l|l|l}
\toprule
\textbf{Pronom} & \textbf{Affirmative} & \textbf{Négative} & \textbf{Interrogative} \\
\midrule
I & I work & I do not (don’t) work & Do I work? \\
You & You work & You don’t work & Do you work? \\
He/She/It & He works & He doesn’t work & Does he work? \\
We & We work & We don’t work & Do we work? \\
You & You work & You don’t work & Do you work? \\
They & They work & They don’t work & Do they work? \\
\bottomrule
\end{tabular}
\end{center}


\subsection*{b. Uses Cases}


\begin{center}
\renewcommand{\arraystretch}{1.3}
\begin{tabular}{p{6cm}|p{6cm}|p{5cm}}
\toprule
\textbf{Use Case} & \textbf{Explanation} & \textbf{Example} \\
\midrule
\textbf{General facts and situations} & Describes facts that are always true. & \textit{The sun rises in the east.} \\
\textbf{Regular actions and routines} & Describes actions that happen regularly. & \textit{I go to the gym on Mondays.} \\
\textbf{With adverbs of frequency} & Shows how often an action occurs. & \textit{She always drinks tea in the morning.} \\
\textbf{Universal or scientific truths} & Expresses scientific or general truths. & \textit{Water boils at 100°C.} \\
\textbf{Timetables and scheduled events} & Describes planned or scheduled events. & \textit{The train leaves at 9 a.m.} \\
\bottomrule
\end{tabular}
\end{center}


\subsubsection*{Remarque : Adverbs of Frequency}

Use the \textbf{Present Simple} when these words appear in a sentence:

\begin{itemize}
    \item always, usually, often, sometimes, rarely, never
    \item every day
    \item on Mondays, on Fridays, etc.
\end{itemize}




\subsection*{c. Common verbes}

\subsection*{Verbe \textbf{to be}}
\begin{center}
\renewcommand{\arraystretch}{1.3}
\begin{tabular}{l|l|l|l}
\toprule
\textbf{Pronom} & \textbf{Affirmative} & \textbf{Négative} & \textbf{Interrogative} \\
\midrule
I & I am & I am not & Am I? \\
You & You are & You are not (aren’t) & Are you? \\
He/She/It & He is & He is not (isn’t) & Is he? \\
We & We are & We are not (aren’t) & Are we? \\
You & You are & You are not (aren’t) & Are you? \\
They & They are & They are not (aren’t) & Are they? \\
\bottomrule
\end{tabular}
\end{center}

\subsection*{Verbe \textbf{to have}}
\begin{center}
\renewcommand{\arraystretch}{1.3}
\begin{tabular}{l|l|l|l}
\toprule
\textbf{Pronom} & \textbf{Affirmative} & \textbf{Négative} & \textbf{Interrogative} \\
\midrule
I & I have & I do not have & Do I have? \\
You & You have & You do not have & Do you have? \\
He/She/It & He has & He does not have & Does he have? \\
We & We have & We do not have & Do we have? \\
You & You have & You do not have & Do you have? \\
They & They have & They do not have & Do they have? \\
\bottomrule
\end{tabular}
\end{center}


\section{Present Continuous (Progressive)}


\subsection*{a. Formulation}

\paragraph{Structure}
\[
\text{Subject} + \text{to be (am / is / are)} + \text{verb} + \text{-ing}
\]

\textbf{Examples:}
\begin{itemize}
  \item I am studying English.
  \item She is watching TV.
  \item They are working on a new project.
\end{itemize}

\paragraph{Negative form}
\[
\text{Subject} + \text{am / is / are not} + \text{verb} + \text{-ing}
\]
\textit{Examples:}  
He isn’t listening. / We aren’t playing football.

\paragraph{Interrogative form}
\[
\text{Am / Is / Are} + \text{subject} + \text{verb} + \text{-ing} \, ?
\]
\textit{Examples:}  
Are you coming? / Is she reading now?

\subsection*{b. Uses of the Present Continuous}

\begin{itemize}
  \item \textbf{Actions happening now:}  
  \textit{Example:} She is talking on the phone right now.
  \item \textbf{Ongoing projects or activities:}  
  \textit{Example:} We are building a new website this month.
  \item \textbf{Temporary situations:}  
  \textit{Example:} I am living with my parents for a few weeks.
  \item \textbf{Changing or developing situations (trends, evolution):}  
  \textit{Example:} The climate is getting warmer.
  \item \textbf{Future plans or arrangements (near future):}  
  \textit{Example:} I’m meeting her soon.
\end{itemize}

\subsection*{c. TOEIC Tips}

\subsubsection*{- With time adverbs}

The Present Continuous is almost always used with time adverbs.

\textbf{Common time adverbs:}  
currently, at the moment, this year, this week, today, still, these days, now, meanwhile, right now.

\textbf{Examples:}
\begin{itemize}
  \item She is studying for her exams at the moment.
  \item We are planning a trip to Japan this year.
  \item He is still working on the project right now.
\end{itemize}

\subsubsection*{- With \textit{always} to express criticism}

Normally, \textit{always} is used with the Present Simple.  
However, when expressing irritation, annoyance, or a negative emotion, we use the Present Continuous.

\textbf{Examples:}
\begin{itemize}
  \item He always loses his keys. \hfill (Present Simple — it’s habitual)
  \item He is always losing his keys! \hfill (Present Continuous — it’s annoying!)
\end{itemize}


\section{Present Perfect Simple}

\subsection*{a. Définition}
Le \textbf{Present Perfect} relie le passé au présent. Il s'utilise pour :
\begin{itemize}
    \item Une action passée avec un lien sur le présent : \textit{I have lost my keys.}
    \item Une expérience de vie (moment non précisé) : \textit{She has visited London.}
    \item Une action commencée dans le passé et qui continue : \textit{I have lived here for 5 years.}
    \item Une action récente avec résultat visible : \textit{He has just finished his report.}
\end{itemize}

\subsection*{b. Formation}
\textbf{Structure :} Sujet + have/has + participe passé (V-ed ou 3e colonne irrégulière)  
\begin{itemize}
    \item Affirmative : \textit{I have worked. / She has finished.}
    \item Négative : \textit{I haven’t seen that movie. / He hasn’t called me yet.}
    \item Interrogative : \textit{Have you ever been to New York? / Has he finished the project?}
\end{itemize}

\subsection*{c. Mots-clés}
\begin{itemize}
    \item \textbf{ever} : déjà, \textbf{never} : jamais
    \item \textbf{just} : venir de, \textbf{already} : déjà
    \item \textbf{yet} : encore / pas encore
    \item \textbf{for / since} : depuis
    \item \textbf{recently / lately} : récemment
\end{itemize}

\subsection*{d. Différence avec le Past Simple}
\begin{tabular}{|l|c|c|}
\hline
Critère & Past Simple & Present Perfect \\
\hline
Moment précis & Oui & Non \\
Lien avec le présent & Non & Oui \\
Mots-clés & yesterday, last week & ever, never, just, already, yet, for, since \\
\hline
\end{tabular}

Exemples :
\begin{itemize}
    \item Past Simple : \textit{I saw that movie yesterday.}
    \item Present Perfect : \textit{I have seen that movie.}
\end{itemize}

\subsection*{e. Astuces TOEIC}
\begin{itemize}
    \item Souvent avec \textit{ever, never, just, yet}.
    \item Différence Present Perfect / Past Simple est souvent testée.
    \item Toujours utiliser Present Perfect avec \textit{since / for}.
\end{itemize}

\subsection*{f. Exemples TOEIC}
\begin{enumerate}
    \item I have already completed the report.
    \item Have you ever attended a business meeting in English?
    \item The manager hasn’t replied to my email yet.
    \item We have been working on this project for three months.
    \item He has just received the contract.
\end{enumerate}



\section{Present Perfect Continuous}


\subsection*{a. Définition}
Le \textbf{Present Perfect Continuous} exprime :
\begin{itemize}
    \item Une action commencée dans le passé et qui continue : \textit{I have been working here for 5 years.}
    \item Une action récente avec résultat visible : \textit{She has been running, that’s why she is tired.}
    \item La durée ou répétition d’une action : \textit{We have been studying English all morning.}
\end{itemize}

\subsection*{b. Formation}
\textbf{Structure :} Sujet + have/has + been + verbe-ing
\begin{itemize}
    \item Affirmative : \textit{I have been reading for two hours. / She has been working since Monday.}
    \item Négative : \textit{I haven’t been sleeping well lately. / He hasn’t been feeling well.}
    \item Interrogative : \textit{Have you been waiting long? / Has she been studying English?}
\end{itemize}

\subsection*{c. Mots-clés}
\begin{itemize}
    \item \textbf{for / since} : depuis
    \item \textbf{lately / recently} : récemment
    \item \textbf{all day / all morning / all week} : toute la journée / toute la semaine
\end{itemize}

\subsection*{d. Différence avec le Present Perfect Simple}
\begin{tabular}{|l|c|c|}
\hline
Critère & Present Perfect Simple & Present Perfect Continuous \\
\hline
Focus & Résultat de l’action & Durée / action en cours \\
Exemple & I have read the book & I have been reading the book \\
Mots-clés & ever, never, already, yet & for, since, lately, all day \\
\hline
\end{tabular}

\subsection*{e. Exemples TOEIC}
\begin{enumerate}
    \item I have been working on this report since morning.
    \item Have you been waiting for the manager long?
    \item She has been traveling a lot recently.
    \item They haven’t been attending meetings lately.
    \item We have been discussing the contract all afternoon.
\end{enumerate}


\section{Past Simple}

\subsection*{a. Formulation}

The \textbf{Past Simple tense} is used to describe actions that started and finished in the past.  
Regular verbs are formed by adding \textbf{-ed} to the base verb.  
Irregular verbs have specific forms that must be memorized.

\[
\left\{
\begin{array}{ll}
\textbf{Affirmative:} & \text{Subject + Verb (past form) + ...} \\
\textbf{Negative:} & \text{Subject + did not (didn't) + Base Verb + ...} \\
\textbf{Interrogative:} & \text{Did + Subject + Base Verb + ...?}
\end{array}
\right.
\]

\begin{tcolorbox}[colback=gray!20,colframe=blue,title=\textbf{Examples}]
\begin{itemize}
    \item \textbf{Affirmative:} She \textbf{worked} yesterday.  
    \item \textbf{Negative:} She \textbf{did not work} yesterday.  
    \item \textbf{Interrogative:} \textbf{Did she work} yesterday?  
\end{itemize}
\end{tcolorbox}

\subsection*{b. Use Cases}

\begin{table}[h]
\centering
\renewcommand{\arraystretch}{1.3}
\begin{tabular}{|p{4cm}|p{6cm}|p{5cm}|}
\hline
\textbf{Use Case} & \textbf{Explanation} & \textbf{Example} \\ \hline

\textbf{Completed past actions} & Describes actions that happened and ended in the past. & I \textbf{visited} Paris last year. \\ \hline

\textbf{Past habits (no longer true)} & Expresses past routines or repeated actions. & She \textbf{walked} to school every day. \\ \hline

\textbf{Sequence of events} & Used to tell a series of completed actions. & He \textbf{entered}, \textbf{sat down}, and \textbf{started} reading. \\ \hline

\textbf{Specific time expressions} & Often used with time markers such as: yesterday, ago, last week, in 2010, etc. & They \textbf{met} two days ago. \\ \hline

\end{tabular}
\caption{Past Simple – Use Cases}
\end{table}

\subsection*{c. Common Irregular Verbs (TOEIC Focus)}

\newpage 
\vspace{1cm}
\begin{table}[h]
\centering
\renewcommand{\arraystretch}{1.3}
\begin{tabular}{|l|l|l|}
\hline
\textbf{Base Form} & \textbf{Past Simple} & \textbf{Meaning} \\ \hline
be & was / were & être \\ \hline
begin & began & commencer \\ \hline
buy & bought & acheter \\ \hline
break & broke & casser \\ \hline 
bring & brought & apporter \\ \hline
build & built & apporter \\ \hline
come & came & venir \\ \hline
catch & caught & attraper \\ \hline 
cost & cost & couter \\ \hline
cut & cut & couper \\ \hline 
do & did & faire \\ \hline
eat & ate & manger \\ \hline 
feel & felt & sentir \\ \hline 
find & found & trouver \\ \hline  
get & got & obtenir \\ \hline
go & went & aller \\ \hline
give & gave & donner \\ \hline 
have & had & avoir \\ \hline
know & knew & savoir \\ \hline 
leave & left & laisser / partir \\ \hline 
make & made & fabriquer / faire \\ \hline
meet & met & rencontrer \\ \hline 
run & rab & courir \\ \hline 
say & said & dire \\ \hline
see & saw & voir \\ \hline
sell & sold & vendre \\ \hline 
send & sent & envoyer \\ \hline 
sit & sat & étre assis \\ \hline 
stand & stood & étre debout \\ \hline 
take & took & prendre \\ \hline
teach & taught & enseigner \\ \hline 
tell & told & dire \\ \hline 
think & thought & penser \\ \hline 
write & wrote & écrire \\ \hline
fall & fell & tomber (fell of ) \\ \hline  
\end{tabular}
% \caption{Common Irregular Verbs in the TOEIC}
\end{table}


\subsection*{d. Tips for TOEIC}

\begin{itemize}
    \item The \textbf{Past Simple} is often used with clear time references (\textit{yesterday, last week, two days ago, in 2020}).  
    \item Avoid confusion with the \textbf{Present Perfect} — if the time is mentioned explicitly, use the \textbf{Past Simple}.  
    \item In spoken English, contractions are very common:  
    \textit{did not → didn’t}, \textit{was not → wasn’t}, \textit{were not → weren’t}.  
    \item Some irregular verbs change vowels only: \textit{sing → sang}, \textit{drink → drank}, \textit{begin → began}.  
\end{itemize}

\newpage
\section{Past Continuous}

\subsection*{a. Formulation}

\[
\text{Subject + was/were + V-ing}
\]

\begin{center}
\begin{tabular}{|c|c|c|c|}
\hline
\textbf{Subject} & \textbf{Auxiliary} & \textbf{Verb} & \textbf{Example} \\
\hline
I / He / She / It & was & V-ing & I was working. \\
You / We / They & were & V-ing & They were studying. \\
\hline
\end{tabular}
\end{center}

\textbf{Negative:} Subject + was/were \textbf{not} + V-ing  
\textit{Example:} I wasn’t sleeping. / They weren’t listening.

\textbf{Interrogative:} Was/Were + Subject + V-ing ?  
\textit{Example:} Was he reading? / Were you playing?


\subsection*{b. Uses Cases}


\begin{center}
\begin{tabular}{|p{4cm}|p{6cm}|p{5cm}|}
\hline
\textbf{Use Case} & \textbf{Explanation} & \textbf{Example} \\
\hline
Action in progress in the past (specific) & To describe an action happening at a specific moment in the past. & At 10 PM yesterday, I was eating dinner. \\
\hline
Interrupted action & One action was ongoing when another (shorter) action happened. & I was studying when Mohamed called. \\
\hline
Two actions in parallel & Two long actions happening at the same time. & I was doing my homework while my family was watching TV. \\
\hline
Setting a scene & To describe the background or context in a story. & The sun was shining, and the birds were singing. \\
\hline
\end{tabular}
\end{center}


\vspace{1cm}
\subsection*{c. Tips TOEIC}

\begin{center}
\begin{tabular}{|p{4cm}|p{6cm}|p{5cm}|}
\hline
\textbf{Tip} & \textbf{Explanation} & \textbf{Example} \\
\hline
After \textbf{while} → Past Continuous & While introduces the longer or ongoing action. & While I was cooking, he was setting the table. \\
\hline
After \textbf{when} → Past Simple & When introduces the short, interrupting action. & I was cooking when he arrived. \\
\hline
Difference: Past Simple vs Past Continuous & Past Simple → completed action.Past Continuous → ongoing action. & I ate dinner at 8 PM. (finished), I was eating dinner at 8 PM. (in progress) \\
\hline
\end{tabular}
\end{center}

\newpage
\section{Past Perfect}

\subsection*{a. Formulation}

\[
\text{Subject + had + past participle (V3)}
\]

\begin{itemize}
  \item \textbf{Affirmative:} She had finished her work before the meeting started.
  \item \textbf{Negative:} She \textit{had not finished} her work before the meeting.
  \item \textbf{Interrogative:} \textit{Had she finished} her work before the meeting?
\end{itemize}


\subsection*{b. Uses}

\begin{center}
\renewcommand{\arraystretch}{1.4}
\begin{tabular}{|p{4cm}|p{6cm}|p{6cm}|}
\hline
\textbf{Use Case} & \textbf{Explanation} & \textbf{Example Sentence} \\
\hline
sequence of events in the past & To indicate what happened first in a sequence & By the time we arrived, the movie had already started. \\
\hline
Action completed before another past event & To show that an action was finished before another action in the past & She had left the office before her manager arrived. \\
\hline 
\end{tabular}
\end{center}

\textbf{Examples :}

\begin{itemize}
    \item I had brushed my teeth before I went to bed.
    \item After I had studied, I watched TV.
    \item When I arrived to the train station, the traun had left.
\end{itemize}


\subsection*{c. Tips for TOEIC}

\begin{itemize}
  \item \textbf{Signal words:} by the time, before, after, as soon as,already, just.  
  \item TOEIC tip: look for a past event mentioned in the sentence — the Past Perfect often appears to show \textbf{what happened first}.  
  \item Difference with Past Simple:
        \begin{itemize}
            \item Past Simple → action completed in the past.  
            \item Past Perfect → action completed \textbf{before another past action}.  
        \end{itemize}
  \item Common pattern: “By the time + past simple, past perfect”.  
        Example: \textit{By the time she arrived, we had finished the meeting.}
\end{itemize}




\newpage
\section{Past Perfect Continuous}

\subsection*{a. Formulation}

\[
\text{Subject + had been + verb(ing)}
\]

\begin{itemize}
  \item \textbf{Affirmative:} She had been working all day.
  \item \textbf{Negative:} She \textit{had not been working} all day.
  \item \textbf{Interrogative:} \textit{Had she been working} all day?
\end{itemize}

\bigskip



\subsection*{b. Uses}

\begin{center}
\renewcommand{\arraystretch}{1.4}
\begin{tabular}{|p{4cm}|p{6cm}|p{6cm}|}
\hline
\textbf{Use Case} & \textbf{Explanation} & \textbf{Example Sentence} \\
\hline
Duration before another past event & To describe an action that continued up to another event in the past & She had been studying for two hours before the test started. \\
\hline
Cause of a past situation & To explain the reason for a past state or feeling & He was tired because he had been running. \\
\hline
Recently completed past actions & To show an action that just finished before another event & It had been raining, so the streets were wet. \\
\hline
\end{tabular}
\end{center}


\subsection*{C. Tips for TOEIC}

\begin{itemize}
  \item \textbf{Signal words:} before, for, since, when, by the time.
  \item Often used to show the \textbf{duration} of an action before another past event.  
        Example: \textit{They had been waiting for an hour before the bus arrived.}
  \item  Difference with \textbf{Past Perfect}:  
        \textit{Past Perfect Continuous} → insist on the \textbf{duration/process}.  
        \textit{Past Perfect} → insist on the \textbf{completion/result}.

\end{itemize}




\chapter{Comparison of Adjectives}

\begin{center}
\begin{tabular}{@{} l l l l @{}} 
\toprule
\textbf{Type} & \textbf{Structure} & \textbf{Example} & \textbf{Note} \\ 
\midrule
Equality & \texttt{as + adj + as} & as big as & expresses equality \\ 
Comparative (short adj) & \texttt{adj + -er + than} & taller than & e.g. big $\rightarrow$ bigger \\ 
Comparative (long adj) & \texttt{more + adj + than} & more beautiful than & positive comparison \\ 
Comparative (negative) & \texttt{less + adj + than} & less interesting than & negative comparison \\ 
Superlative (short adj) & \texttt{the + adj + -est} & the smallest &  \\ 
Superlative (long adj) & \texttt{the most + adj} & the most intelligent & positive superlative \\ 
Superlative (negative) & \texttt{the least + adj} & the least expensive & negative superlative \\ 
\bottomrule
\end{tabular}
\end{center}


\begin{center}
\textbf{RQ: Exceptions of Adjective Comparison Form}

\vspace{0.5cm}

\begin{tabular}{@{} l l l l @{}} 
\toprule
\textbf{Adjective} & \textbf{Comparative} & \textbf{Superlative} & \textbf{Note / Meaning} \\ 
\midrule
good & better & best & irregular form (bon → meilleur) \\ 
bad & worse & worst & irregular form (mauvais → pire) \\ 
far & farther / further & farthest / furthest & both correct (distance or degree) \\ 
little & less & least & quantity or size \\ 
much / many & more & most & quantity comparison \\ 
\bottomrule
\end{tabular}
\end{center}



\begin{center}
\renewcommand{\arraystretch}{1.4}
\small
\begin{tabular}{|p{2.8cm}|p{3cm}|p{4.5cm}|p{4.5cm}|}
\hline
\textbf{Word Type} & \textbf{Common Suffixes} & \textbf{Explanation / Meaning} & \textbf{Examples} \\
\hline
\textbf{Noun} 
& -tion, -sion, -ment, -ness, -ity, -er, -or, -ist, -ance, -ence, -ship, -hood 
& Action, person, state, or quality 
& creation, happiness, actor, friendship, childhood \\
\hline
\textbf{Verb} 
& -ize/-ise, -ify, -en, -ate 
& To make, cause, or become 
& organize, simplify, widen, celebrate \\
\hline
\textbf{Verb used as Noun (Gerund)} 
& -ing 
& Verb form acting as a noun (subject or object) 
& \textit{Reading is fun.} / \textit{I enjoy swimming.} \\
\hline
\textbf{Adjective} 
& -able/-ible, -ous/-ious, -ful, -less, -al, -ive, -ic/-ical, -y 
& Describes quality or characteristic 
& possible, curious, beautiful, hopeless, natural, creative, poetic, sleepy \\
\hline
\textbf{Verb used as Adjective (Participle)} 
& -ing (present), -ed (past) 
& Present participle = causing feeling / Past participle = feeling or result 
& \textit{The movie is boring. / I am bored.} \\
\hline
\textbf{Adverb} 
& -ly, -ward/-wards, -wise 
& Describes how, where, or in what manner 
& quickly, forward, otherwise \\
\hline
\end{tabular}
\end{center}

\chapter{Conditions:}


\begin{table}[h!]
\centering
\renewcommand{\arraystretch}{1.3}
\begin{tabular}{|c|p{3.3cm}|p{4.1cm}|p{4.5cm}|p{3.5cm}|}
\hline
\textbf{N°} & 
\textbf{Structure} &
\textbf{Usage (English)} &
\textbf{Example} &
\textbf{Remarks} \\
\hline

0 &
If + Present Simple, Present Simple &
General truths, scientific facts, habits &
\textit{If you heat ice, it melts.} &
You can use \textbf{when} instead of \textit{if} when the result is certain. \\
\hline

1 &
If + Present Simple, Will + infinitive &
Real and possible situations in the future &
\textit{If it rains, we will stay home.} &
\textbf{unless} = \textit{if … not}. \\
\hline

2 &
If + Past Simple, Would + infinitive &
Unreal or imaginary situations in the present or future &
\textit{If I had more money, I would travel.} &
Use \textbf{were} instead of \textit{was} after I/he/she;  
\textbf{might/could} can replace \textit{would}. \\
\hline

3 &
If + Past Perfect, Would have + past participle &
Unreal situations in the past / regrets &
\textit{If you had called, I would have come.} &
--- \\
\hline

Mixed &
If + Past Perfect, Would + infinitive  &
Past condition → present result &
\textit{If I had studied, I would be more confident now.} &
--- \\
\hline

\end{tabular}
\caption{Summary of English Conditional Forms}
\end{table}





\chapter{Difference between \textbf{GOING TO} and \textbf{WILL}}


\textbf{Both "will" and "going to" are used to express the future, but they are not used in the same situations.}

\begin{center}
\renewcommand{\arraystretch}{1.4}
\begin{tabular}{|p{3cm}|p{5cm}|p{5cm}|}
\hline
\textbf{Usage} & \textbf{WILL} & \textbf{GOING TO} \\
\hline
\textbf{1. Decision moment} & Used for decisions made at the moment of speaking. & Used for decisions already made before the moment of speaking. \\
\hline
\textbf{Example} & I'm thirsty. I \textbf{will} buy a bottle of water. & I bought a bottle of water because I \textbf{was going to} get thirsty. \\
\hline
\textbf{2. Prediction (based on opinion)} & Used to express what we \textbf{think or believe} will happen. & Used when there is \textbf{evidence} that something is going to happen. \\
\hline
\textbf{Example} & I think it \textbf{will} rain tomorrow. & Look at those clouds! It’s \textbf{going to} rain soon. \\
\hline
\textbf{3. Promise / Offer / Request} & Used for promises, offers, or requests. & Not used in this context. \\
\hline
\textbf{Example} & I \textbf{will} help you with your report. &  (Not used) \\
\hline
\textbf{4. Planned future action} & Not used for pre-planned actions. & Used for planned or intended future actions. \\
\hline
\textbf{Example} &  (Not natural) & We are \textbf{going to} visit Paris next week. \\
\hline
\textbf{5. Spontaneous speech} & Used when we decide or react instantly. & Used when the plan already exists. \\
\hline
\textbf{Example} & Oh, the phone is ringing — I \textbf{will} answer it! & I’m \textbf{going to} call him later today. \\
\hline
\end{tabular}
\end{center}

\vspace{0.3cm}
\textbf{Summary:}
\begin{itemize}
    \item \textbf{WILL} → instant decisions, promises, beliefs, offers.
    \item \textbf{GOING TO} → plans, intentions, and predictions with evidence.
\end{itemize}

\vspace{0.3cm}
\textbf{Examples in context:}
\begin{itemize}
    \item I think he \textbf{will} pass the test. (opinion)
    \item He’s \textbf{going to} study harder next month. (plan)
\end{itemize}


\chapter{Pronouns}


\section{Relative Pronouns}

\subsection*{a. Definition}

A \textbf{relative pronoun} introduces a clause that gives more information about a noun.  
It connects two ideas into one sentence and avoids repetition. Relative clauses describe or identify a person, place, thing, or idea mentioned in the main clause.



\subsection*{Formulation Rule}

\[
\left\{
\begin{array}{ll}
\textbf{Form 1:} & \text{Noun (person / thing) + Relative Pronoun + Clause} \\[6pt]
\textbf{Form 2:} & \text{Noun + (who / which / that / whose / where / whom) + Subject + Verb + [Object]} \\[6pt]
\textbf{Omission Rule:} & \text{If the relative pronoun is the \textit{object} of the clause, it can be omitted.}
\end{array}
\right.
\]

\begin{center}
\renewcommand{\arraystretch}{1.3}
\begin{tabular}{p{3cm}|p{6cm}|p{5cm}}
\toprule
\textbf{Structure} & \textbf{Sentence} & \textbf{Translation} \\
\midrule
\textbf{Form 1} & The man \textbf{who lives next door} is friendly. & L’homme \textbf{qui habite à côté} est gentil. \\
\textbf{Form 2} & The book \textbf{which I read} was interesting. & Le livre \textbf{que j’ai lu} était intéressant. \\
\textbf{Omission rule} & The woman \textbf{(who) you met} is my teacher. & La femme \textbf{que tu as rencontrée} est ma prof. \\
\bottomrule
\end{tabular}
\end{center}


\subsection*{b. Relative Pronouns Table}

\begin{center}
\renewcommand{\arraystretch}{1.3}
\begin{tabular}{p{2.5cm}|p{3cm}|p{4cm}|p{5cm}}
\toprule
\textbf{Pronoun} & \textbf{Refers to} & \textbf{Explanation} & \textbf{Example} \\
\midrule
\textbf{who} & People (subject) & Used for the person performing the action. & The man \textbf{who lives} next door is friendly. \\
\textbf{whom} & People (object) & Used for the person receiving the action (formal). & The woman \textbf{whom we met} is a doctor. \\
\textbf{which} & Things / Animals & Adds information about an object or animal. & The book \textbf{which I bought} is interesting. \\
\textbf{that} & People / Things & Can replace “who” or “which” in informal contexts. & The car \textbf{that I drive} is new. \\
\textbf{whose} & Possession & Indicates ownership or relationship. & The man \textbf{whose car broke down} is my friend. \\
\textbf{where} & Places & Refers to a place. & The city \textbf{where I was born} is beautiful. \\
\bottomrule
\end{tabular}
\end{center}



\subsection*{c. Uses}

\begin{itemize}
    \item \textbf{To connect two phrases (to avoid repetition):}  
    \textit{Ahmed used to be an engineer. Ahmed is now a YouTuber.}  
    → \textbf{Ahmed, who used to be an engineer, is now a YouTuber.}
    
    \item \textbf{To specify or describe:}  
    \textit{According to statistics, people who eat organic food live longer.}  
    \textit{I love books that have drawings in them.}
\end{itemize}


\section{Linking words}

\subsection*{a. Definition}

\textbf{Linking words} (also called \textit{connectors} or \textit{transition words}) are expressions used to \textbf{connect ideas, sentences, or paragraphs}.  
They help to:
\begin{itemize}
    \item make writing more \textbf{coherent} and \textbf{fluent};
    \item show \textbf{logical relationships} between ideas (cause, contrast, addition, etc.);
    \item guide the reader through your reasoning.
\end{itemize}

They are essential in \textbf{essays, reports, formal emails, and presentations}.


\subsection*{b. Table of Common Linking Words}


\begin{center}
\renewcommand{\arraystretch}{1.3}
\begin{tabular}{p{3cm}|p{4cm}|p{3.5cm}|p{6cm}}
\textbf{Word / Expression} & \textbf{Formulation} & \textbf{Use / Function} & \textbf{Example} \\ \hline
\textbf{Although} & Although + subject + verb & Introduces a contrast & \textit{Although it was raining, we went out.} \\ \hline
\textbf{Though} & Though + subject + verb & Contrast (informal or spoken) & \textit{I like it, though it’s expensive.} \\ \hline
\textbf{Even though} & Even though + subject + verb & Strong contrast & \textit{Even though he studied hard, he failed.} \\ \hline
\textbf{Despite} & Despite + noun / V-ing & Contrast & \textit{Despite the rain, we went out.} \\ \hline
\textbf{In spite of} & In spite of + noun / V-ing & Contrast (formal) & \textit{In spite of being tired, she kept working.} \\ \hline
\textbf{However} & However + clause & Contrast between two ideas & \textit{It was raining. However, we went out.} \\ \hline
\textbf{Therefore} & Therefore + clause & Shows a consequence or conclusion & \textit{He didn’t study; therefore, he failed.} \\ \hline
\textbf{Thus} & Thus + clause & Formal equivalent of "therefore" & \textit{The data were incomplete; thus, the result is uncertain.} \\ \hline
\textbf{Consequently} & Consequently + clause & Logical consequence & \textit{Prices rose; consequently, demand fell.} \\ \hline
\textbf{Moreover} & Moreover + clause & Adds an idea (formal) & \textit{She’s smart. Moreover, she’s very kind.} \\ \hline
\textbf{Furthermore} & Furthermore + clause & Adds an idea (formal) & \textit{The plan is expensive. Furthermore, it’s risky.} \\ \hline
\textbf{In addition} & In addition + clause / noun & Adds information or argument & \textit{In addition, the project will create jobs.} \\ \hline
\textbf{On the other hand} & On the other hand + clause & Introduces an opposite opinion & \textit{He likes big cities. On the other hand, I prefer the countryside.} \\ \hline
\textbf{Whereas} & Whereas + subject + verb & Direct contrast between two facts & \textit{Whereas I prefer tea, she prefers coffee.} \\ \hline
\textbf{For example} & For example + noun / clause & Introduces an example & \textit{Some animals, for example lions, hunt in groups.} \\ \hline
\textbf{For instance} & For instance + noun / clause & Example (same as "for example") & \textit{Many countries, for instance Japan, use robots.} \\ \hline
\textbf{In conclusion} & In conclusion + clause & Used to conclude a text & \textit{In conclusion, education is key to success.} \\ \hline
\textbf{To sum up} & To sum up + clause & Summarizes the main ideas & \textit{To sum up, we must act now.} \\ \hline
\textbf{Besides} & Besides + noun / clause & Adds a secondary idea or reason & \textit{I don’t want to go. Besides, it’s too late.} \\ \hline
\textbf{Nevertheless} & Nevertheless + clause & Contrast (formal, like "however") & \textit{It was hard; nevertheless, he succeeded.} \\ \hline
\textbf{As a result} & As a result + clause & Expresses a consequence & \textit{The company lost money; as a result, they cut jobs.} \\ \hline
\textbf{Hence} & Hence + clause / noun & Formal, shows consequence & \textit{He’s the CEO, hence the high salary.} \\ \hline
\end{tabular}
\end{center}


\subsection*{c. Tips }

\begin{itemize}
    \item \textbf{Although / though / even though} → followed by a \textbf{full sentence}. \\
    Example: \textit{Although it was late, he kept working.}
    \item \textbf{Despite / in spite of} → followed by a \textbf{noun or -ing form}. \\
    Example: \textit{Despite being tired, he worked.}
    \item \textbf{However, therefore, moreover} → usually placed \textbf{after a period or semicolon}. \\
    Example: \textit{He was tired; however, he continued.}
    \item Don’t confuse \textit{although} and \textit{despite} – they have different grammatical structures.
\end{itemize}



% \section*{D. Usage Tips}

% \begin{itemize}
%     \item Use linking words to \textbf{structure your writing}: introduction, development, and conclusion.  
%     \item \textbf{Vary} your connectors – don’t always repeat "because" or "so".  
%     \item Choose the correct type of connector for the relationship:
%     \begin{itemize}
%         \item \textbf{Addition} → moreover, furthermore, in addition
%         \item \textbf{Contrast} → however, although, despite
%         \item \textbf{Cause / Effect} → because, therefore, as a result
%         \item \textbf{Example} → for example, for instance
%         \item \textbf{Conclusion} → in conclusion, to sum up
%     \end{itemize}
% \end{itemize}



\section{Clauses of Result}

\subsection*{a. Definition}
A \textbf{result clause} shows the \textit{outcome or consequence} of an action or situation mentioned in the main clause.  
It answers the question: \textit{"What happened as a result?"}  
We often use expressions such as \textbf{so...that}, \textbf{such...that}, \textbf{since}, \textbf{now that}, and \textbf{as long as} to introduce result or reason clauses.

\subsection*{b. Result Clauses}

\begin{center}
\begin{tabular}{|p{2cm}|p{5cm}|p{4.5cm}|p{5.5cm}|}
\hline
\textbf{Clause Type} & \textbf{Form} & \textbf{Use} & \textbf{Example} \\
\hline
\textbf{SO...THAT} & \textit{Subject + be + so + adjective + that + clause} & Used to show a result caused by the degree of an adjective. & Mike is \textbf{so clever that} he always gets good grades. \\
\hline
\textbf{SO...THAT} & \textit{Subject + verb + so + adverb + that + clause} & Used to show a result caused by the degree of an adverb. & He walked \textbf{so slowly that} he missed the bus. \\
\hline
\textbf{SUCH... THAT} & \textit{Subject + be + such (+ a/an) + adjective + noun + that + clause} & Used to show a result caused by the intensity or quality of a noun phrase. \newline (Use \textit{a/an} with singular countable nouns.) & Dora is \textbf{such a good teacher that} everybody admires her.  \\
\hline
\textbf{SINCE} & \textit{Since + subordinate clause, + main clause} & Introduces a reason for the action or situation in the main clause. & \textbf{Since} you speak perfect English, I want you to translate for me. \newline We decided to stop, \textbf{since} it was dark. \\
\hline
\textbf{NOW THAT} & \textit{Now that + subordinate clause, + main clause} & Introduces a new or recent condition that explains the main clause. & \textbf{Now that} she is sixty, she may decide to stop working. \newline I’ll go home, \textbf{now that} the party is over. \\
\hline
\textbf{AS LONG AS} & \textit{As long as + subordinate clause, + main clause} & Expresses a condition necessary for the action or situation in the main clause. & \textbf{As long as} Fred agrees to help me, I’ll help him. \newline You can drink the water, \textbf{as long as} you boil it first. \\
\hline
\end{tabular}
\end{center}

\subsection*{Summary}
\begin{itemize}
    \item \textbf{so...that} → result caused by an adjective or adverb.  
    \item \textbf{such...that} → result caused by an adjective + noun phrase.  
    \item \textbf{since} → introduces a reason.  
    \item \textbf{now that} → introduces a new condition.  
    \item \textbf{as long as} → expresses a condition or requirement.
\end{itemize}



\chapter{Modals}

\section{Modal Verbs}

\subsection*{a. Definition}

A \textbf{modal verb} is an auxiliary verb that expresses the speaker’s attitude or the “mode” of the main verb.  
It indicates possibility, ability, permission, necessity, or advice.  
Modal verbs are always followed by the \textbf{base form} of the main verb (without “to”).

\[
\left\{
\begin{array}{ll}
\textbf{Affirmative:} & \text{Subject + Modal Verb + Base Verb} \\
\textbf{Negative:} & \text{Subject + Modal Verb + not + Base Verb} \\
\textbf{Interrogative:} & \text{Modal Verb + Subject + Base Verb?}
\end{array}
\right.
\]

\begin{tcolorbox}[colback=lightgray!40,colframe=blue,title=\textbf{Examples}]
can, could, may, might, shall, should, will, would, must, ought to, had better
\end{tcolorbox}


\subsection*{b. Modals}

\subsubsection*{Capacity / Suggestion / Advice :}

\begin{center}
\renewcommand{\arraystretch}{1.3}
\begin{tabular}{p{2.2cm}|p{5cm}|p{5cm}|p{1cm}|p{1cm}}
\toprule
\textbf{Modal Verb} & \textbf{Explanation} & \textbf{Sense} & \textbf{Neg} & \textbf{Interro} \\
\midrule
\textbf{can} & Expresses present ability or possibility. & Ability  & yes & yes \\
\textbf{could} & Expresses past ability or polite suggestion. & Past ability  & yes & yes \\
\textbf{shall} & Expresses suggestion or future intention (formal). & Suggestion  & yes & yes \\
\textbf{should} & Expresses advice, recommendation, or mild obligation. & Advice  & yes & yes \\
\textbf{ought to} & Expresses moral duty or strong advice (similar to “should”). & Advice  & no & no \\
\textbf{had better} & Expresses strong advice or warning (informal). & Strong advice / Warning & yes & no \\
\bottomrule
\end{tabular}
\end{center}

\subsubsection*{Examples :}

\begin{itemize}
    \item \textbf{Ability:} I \textbf{can} speak three languages.  
    \item \textbf{Past ability:} When I was young, I \textbf{could} run fast.  
    \item \textbf{Suggestion:} \textbf{Shall} we go for a walk?  
    \item \textbf{Advice:} You \textbf{should} see a doctor.  
    \item \textbf{Strong advice:} You \textbf{had better} study for the test.  
\end{itemize}


\subsubsection*{Permission / Request}

\begin{table}[h]
\centering
\renewcommand{\arraystretch}{1.3}
\begin{tabular}{|l|p{5cm}|p{3cm}|p{5cm}|}
\hline
\textbf{Modal Verb} & \textbf{Explanation} & \textbf{Sense} & \textbf{Example} \\ \hline

\textbf{can} & Used to ask or give permission in informal situations. & Informal permission / request & Can I leave early today? \\ \hline

\textbf{could} & Polite or tentative form of “can” when asking permission. & Polite request / permission & Could I borrow your pen, please? \\ \hline

\textbf{will} & Used to make a direct request or express willingness. & Request & Will you help me with my homework? \\ \hline

\textbf{would} & More polite and formal than “will”. Often used in polite offers or requests. & Polite request  & Would you like a cup of tea? \\ \hline

\textbf{may} & Formal way to ask or give permission. & Formal permission & May I come in, sir? \\ \hline

\textbf{might} & Slightly more hesitant or uncertain than “may”. & Polite hesitant permission & Might I ask you a question? \\ \hline

\end{tabular}
% \caption{Modal Verbs for Permission and Demand}
\end{table}


\subsubsection*{Obligation / Necessity :}

\begin{table}[h]
\centering
\renewcommand{\arraystretch}{1.3}
\begin{tabular}{|l|p{5cm}|p{3cm}|p{5cm}|}
\hline
\textbf{Modal Verb} & \textbf{Explanation} & \textbf{Sense} & \textbf{Example} \\ \hline

\textbf{must} & Expresses a strong obligation or necessity imposed by the speaker. & Strong obligation / necessity & You \textbf{must} wear a seatbelt in the car. \\ \hline

\textbf{must not} & Expresses prohibition — something is not allowed. & Prohibition / strong negative obligation & You \textbf{must not} smoke in the hospital. \\ \hline

\textbf{have to} & Expresses an external obligation, rule, or requirement. & External obligation / necessity & I \textbf{have to} submit the report by Monday. \\ \hline

\textbf{does not have to} & Expresses a lack of necessity — something is not required. & Absence of obligation & She \textbf{does not have to} come if she’s busy. \\ \hline

\textbf{have got to} & Informal form of “have to”, mainly used in spoken English. & Informal necessity / strong obligation & I’ve \textbf{got to} finish this work today. \\ \hline

\end{tabular}
% \caption{Modal Verbs for Obligation and Necessity}
\end{table}

\chapter{Passive voice}


\section{Passive voice}

\subsection*{a. Definition}

The \textbf{Passive Voice} is used when the \textbf{focus is on the action} or the \textbf{receiver of the action}, not on who performs it.  
The subject of the active sentence becomes the \textbf{agent} (introduced by "by") or is omitted if not important.

\medskip
\subsubsection*{Formation:}  
\[
\text{Passive} = \textbf{Be} + \text{Past Participle (P.P.)}
\]
Example:  
\textit{Active:} The chef cooks the meal.  
\textit{Passive:} The meal \textbf{is cooked} (by the chef).

\subsection*{b. Uses of the Passive Voice}

\begin{table}[ht]
\centering
\renewcommand{\arraystretch}{1.4}
\begin{tabular}{|p{3cm}|p{6cm}|p{5cm}|}
\hline
\textbf{Use} & \textbf{Explanation} & \textbf{Example} \\ \hline
When the doer is unknown & The person who performs the action is not known or not important. & My bike \textbf{was stolen} yesterday. \\ \hline
When the doer is obvious & It is clear who did the action, so we don’t need to mention them. & Taxes \textbf{are collected} every year. \\ \hline
When the action is more important than the subject & The focus is on the result, not the person. & A new hospital \textbf{has been built}. \\ \hline
In formal or scientific writing & Used to sound objective or impersonal. & The data \textbf{were analyzed} using Python. \\ \hline
To avoid blame or responsibility & The speaker avoids saying who caused the problem. & A mistake \textbf{was made}. \\ \hline
\end{tabular}
\end{table}

\subsection*{c. Transforming Active to Passive}

To change a sentence from the \textbf{Active Voice} to the \textbf{Passive Voice}:

\begin{enumerate}
    \item Remove or move the subject (agent) to the end of the sentence with “by”.
    \item Place the object of the active sentence as the new subject.
    \item Add the correct form (the same tense of the verb in the active form) of \textbf{Be} + \textbf{Past Participle (P.P.)}.
    % \item Keep the same tense of the verb in the passive form.
\end{enumerate}

\textbf{Example formula:}
\[
\text{Active: Subject + Verb + Object} \Rightarrow \text{Passive: Object + Be + P.P. (+ by + Subject)}
\]

\subsubsection*{Examples in All Tenses :}

\begin{table}[ht]
\centering
\renewcommand{\arraystretch}{1.4}
\begin{tabular}{|p{3cm}|p{5cm}|p{8cm}|}
\hline
\textbf{Tenses} & \textbf{Active Voice} & \textbf{Passive Voice} \\ \hline
\textbf{Present Simple} & Mohamed kills Ahmed. & Ahmed \textbf{is killed} (by Mohamed). \\ \hline
\textbf{Present Continuous} & Mohamed is killing Ahmed. & Ahmed \textbf{is being killed} (by Mohamed). \\ \hline
\textbf{Present Perfect} &  Mohamed has killed Ahmed. & Ahmed \textbf{has been killed} (by Mohamed). \\ \hline
\textbf{Past Simple} & Mohamed killed Ahmed. & Ahmed \textbf{was killed} (by Mohamed). \\ \hline
\textbf{Past Continuous} & Mohamed was killing Ahmed. & Ahmed \textbf{was being killed} (by Mohamed). \\ \hline
\textbf{Past Perfect} & Mohamed had killed Ahmed. & Ahmed \textbf{had been killed} (by Mohamed). \\ \hline
\textbf{Future Simple} & Mohamed will kill Ahmed. & Ahmed \textbf{will be killed} (by Mohamed). \\ \hline
\textbf{Future Continuous} &  Mohamed will be killing Ahmed. & Ahmed \textbf{will be being killed} (by Mohamed). \\ \hline
\textbf{Future Perfect} & Mohamed will have killed Ahmed. & Ahmed \textbf{will have been killed} (by Mohamed). \\ \hline
\end{tabular}
\end{table}





\section{Reported Speech}

\subsection*{a. Definition}
Reported Speech (or Indirect Speech) is used to tell what someone \textbf{said}, \textbf{asked}, or \textbf{told}, without quoting their exact words.  
When we report speech, we often change the \textbf{tense}, \textbf{pronouns}, and \textbf{time expressions} to fit the context.

\textbf{Example:}
\begin{itemize}
    \item Direct: He said, "I am tired."
    \item Reported: He said (that) he was tired.
\end{itemize}

\subsection*{b. Formation: Steps}
\begin{enumerate}
    \item Change the pronouns if necessary (I → he/she, my → his/her, etc.)
    \item Change the tense of the verb (backshift).
    \item Adjust time and place expressions (now → then, today → that day, etc.)
\end{enumerate}

\subsection*{c. Changes of Tenses}
\renewcommand{\arraystretch}{1.3}
\begin{tabular}{|l|l|}
\hline
\textbf{Direct Speech} & \textbf{Reported Speech} \\
\hline
Present Simple $\rightarrow$ Past Simple & "I play football" → He said (that) he played football. \\
\hline
Present Continuous $\rightarrow$ Past Continuous & "I am playing football" → He said (that) he was playing football. \\
\hline
Present Perfect $\rightarrow$ Past Perfect & "I have finished" → He said (that) he had finished. \\
\hline
Past Simple $\rightarrow$ Past Perfect & "I went home" → He said (that) he had gone home. \\
\hline
Will $\rightarrow$ Would & "I will help you" → He said (that) he would help me. \\
\hline
Be going to $\rightarrow$ Was/Were going to & "I am going to travel" → He said (that) he was going to travel. \\
\hline
Can $\rightarrow$ Could & "I can swim" → He said (that) he could swim. \\
\hline
May $\rightarrow$ Might & "I may come" → He said (that) he might come. \\
\hline
Must/Have to $\rightarrow$ Had to & "I must leave" → He said (that) he had to leave. \\
\hline
\end{tabular}

\subsection*{d. Time and Place Changes}
\begin{tabular}{|l|l|}
\hline
\textbf{Direct} & \textbf{Indirect} \\
\hline
today & that day \\
now & then / at that moment \\
yesterday & the day before \\
\ldots days ago & \ldots days before \\
last week & the week before \\
next year & the following year \\
tomorrow & the next day / the following day \\
here & there \\
this & that \\
these & those \\
ago & before / previously \\
tonight & that night \\
\hline
\end{tabular}

\subsection*{e. Examples:}
\begin{itemize}
    \item \textbf{Present Simple → Past Simple:}  
    He said, "I eat pizza." → He said (that) he ate pizza.
    \item \textbf{Present Continuous → Past Continuous:}  
    She said, "I am studying." → She said (that) she was studying.
    \item \textbf{Present Perfect → Past Perfect:}  
    He said, "I have finished my work." → He said (that) he had finished his work.
    \item \textbf{Past Simple → Past Perfect:}  
    They said, "We went to Paris." → They said (that) they had gone to Paris.
    \item \textbf{Future (will) → would:}  
    She said, "I will call you." → She said (that) she would call me.
    \item \textbf{Be going to → was/were going to:}  
    He said, "I am going to start a new job." → He said (that) he was going to start a new job.
    \item \textbf{Can → Could:}  
    He said, "I can help you." → He said (that) he could help me.
    \item \textbf{May → Might:}  
    She said, "I may come tomorrow." → She said (that) she might come the next day.
    \item \textbf{Must → Had to:}  
    He said, "I must leave now." → He said (that) he had to leave then.
\end{itemize}





\section{Reported Questions}

\subsection*{a. Definition}

\textbf{Reported Questions} (also called \textit{Indirect Questions}) are used when we report what someone asked, without quoting their exact words.  
\\[4pt]
\textbf{Example:}
\begin{itemize}
    \item Direct: He asked, “Where is my book?”
    \item Reported: He asked where his book was.
\end{itemize}


\subsection*{b. Formation (Steps)}

To change a direct question into a reported question, follow these steps:

\begin{enumerate}
    \item Remove quotation marks and the question mark.
    \item Change pronouns and verb tenses (backshift of tenses).
    \item Keep the word order as \textbf{subject + verb} (no inversion).
    \item Use \textbf{if} or \textbf{whether} for Yes/No questions.
    \item Keep the question word (\textbf{who, what, where, when, why, how}) for Wh-questions.
\end{enumerate}


\subsection*{c. Types of Reported Questions}

\subsubsection*{1. Yes / No Questions}

Use \textbf{if} or \textbf{whether} to introduce the reported question.

\begin{tabular}{|m{5cm}|m{8cm}|}
\hline
\textbf{Direct Question} & \textbf{Reported Question} \\
\hline
He asked, “Do you like apples?” & He asked if I liked apples. \\
\hline
She asked, “Is he coming?” & She asked whether he was coming. \\
\hline
They asked, “Can you help us?” & They asked if I could help them. \\
\hline
\end{tabular}


\subsubsection*{2. Wh-Questions}

Keep the question word, but use the normal order (\textbf{subject + verb}), not inversion.

\begin{tabular}{|m{5cm}|m{8cm}|}
\hline
\textbf{Direct Question} & \textbf{Reported Question} \\
\hline
He asked, “Where is John?” & He asked where John was. \\
\hline
She asked, “What are you doing?” & She asked what I was doing. \\
\hline
They asked, “Why did you go there?” & They asked why I had gone there. \\
\hline
\end{tabular}



\subsection*{d. Examples :}

\begin{itemize}
    \item “Where do you live?” → He asked where I \textbf{lived.} (Present Simple → Past Simple)
    \item “What are you doing?” → He asked what I \textbf{was doing.} (Present Continuous → Past Continuous)
    \item “Why did you go there?” → He asked why I \textbf{had gone there.} (Past Simple → Past Perfect)
    \item “Have you seen my car?” → He asked if I \textbf{had seen} his car. (Present Perfect → Past Perfect)
    \item “Will you help me?” → He asked if I \textbf{would help him.} (Future → Conditional)
    \item “Can you come?” → He asked if I \textbf{could come.}
    \item “Are you coming?” → He asked if I \textbf{was coming.}
    \item “Where will she go?” → He asked where she \textbf{would go.}
    \item “Do they know you?” → He asked if they \textbf{knew me.}
\end{itemize}

\paragraph{Remark. Indirect Questions (Embedded Questions).}

In an indirect question, we never use subject–auxiliary inversion.  
After introductory expressions such as \textit{I wonder}, \textit{I don't know}, or 
\textit{Could you tell me}, the sentence follows a normal declarative word order:

\[
\text{question word} + \text{subject} + \text{verb}
\]

Example:
\begin{itemize}
    \item Direct question: \textit{How much money does a professor make?}
    \item Indirect question: \textit{I wonder how much money a professor makes.}
\end{itemize}

\textbf{Common introductory expressions for indirect questions:}
\begin{itemize}
    \item I wonder
    \item I don't know / I have no idea
    \item Could you tell me
    \item Do you know
    \item Can you explain
    \item Please tell me
    \item I'd like to know
    \item I'm not sure
    \item Tell me
\end{itemize}

These expressions introduce an embedded clause, which must keep the declarative structure (no inversion).

\chapter{Causative}

\section{Causative Verbs}

\subsection*{a. Definition :}

A \textbf{causative verb} is used when the subject \textit{does not perform the action directly}, but instead \textit{causes someone else to do it}.  
In other words, the subject is responsible for the action being done, even though another person actually performs it.

\subsection*{b. Types :}

\begin{center}
\begin{tabular}{|p{3cm}|p{1cm}|p{3cm}|p{5cm}|p{4cm}|}
\hline
\textbf{Type} & \textbf{Verb} & \textbf{Formation} & \textbf{Meaning} & \textbf{Example} \\
\hline
\textbf{Paying Services} & have & have + something + V3 (past participle) & You pay someone to do the action for you. More formal. The person who does the action is not important. & I \textbf{have my shirts ironed} every week. \\
\cline{2-5}
 & get & get + something + V3 & Same meaning as “have”, but more informal. & I \textbf{get my hair cut} once a month. \\
\hline
\textbf{Ask / Convince / Persuade} & have & have + someone + V1 & You ask someone to do the action for you. The person who does the action is important. & I \textbf{had my assistant send} the email. \\
\cline{2-5}
 & get & get + someone + to + V1 & You try to convince or persuade someone to do the action (harder than “have”). & I \textbf{got him to help} me with the report. \\
\hline
\textbf{Force} & make & make + someone + V1 & You force or oblige someone to do the action. & The teacher \textbf{made the students clean} the classroom. \\
\hline
\textbf{Permission} & let & let + someone + V1 & You allow someone to do the action. & My parents \textbf{let me go} out last night. \\
\hline
\end{tabular}
\end{center}

\subsection*{Summary}
\begin{itemize}
    \item \textbf{have / get + something + V3} → paying for a service.
    \item \textbf{have + someone + V1} → ask someone.
    \item \textbf{get + someone + to + V1} → persuade someone.
    \item \textbf{make + someone + V1} → force someone.
    \item \textbf{let + someone + V1} → allow someone.
\end{itemize}





\chapter{Subjunctive Mood}

\section*{a. Definition }


\textbf{Definition : }

The subjunctive mood expresses actions or states that are \emph{hypothetical}, \emph{desired}, \emph{necessary}, or \emph{contrary to fact}. It is used when we do not state a plain fact but rather a wish, a demand, a recommendation, or an imagined situation.

\vspace{1cm}
\noindent
\textbf{Formation :}

\begin{itemize}
  \item For most verbs: use the \emph{base form} (infinitive without to) after \texttt{that + subject}: \textit{I suggest that he \textbf{study} harder.}
  \item For the verb \textit{to be}: use \textbf{be} for all persons: \textit{It is essential that she \textbf{be} on time.}
  \item In unreal (contrary-to-fact) present situations use \textbf{were} for all subjects: \textit{I wish I \textbf{were} taller.}
\end{itemize}



\section*{b. Verbs that trigger the subjunctive}
Many verbs of suggestion, demand, request, and command are followed by \texttt{that + S + base verb}.
\medskip
\noindent\textbf{Common verbs:} \textit{suggest, recommend, insist, ask, request, demand, command, order, propose.}

\medskip
\noindent\textbf{Examples:}
\begin{enumerate}[nosep]
  \item \textit{I suggest that he \textbf{take} the exam again.}
  \item \textit{She recommended that I \textbf{read} this book.}
  \item \textit{They insisted that he \textbf{be} present.}
  \item \textit{The manager demanded that we \textbf{finish} the report.}
  \item \textit{He asked that she \textbf{come} early.}
  \item \textit{We proposed that the meeting \textbf{be} postponed.}
\end{enumerate}

\section*{c. Adjectives that trigger the subjunctive}
Adjectives expressing necessity, importance, or urgency are often followed by a clause with the subjunctive.
\medskip
\noindent\textbf{Common adjectives:} \textit{important, vital, desirable, necessary, imperative, crucial, essential.}
\medskip
\noindent\textbf{Pattern:} \textit{It is + adjective + that + subject + base verb.}
\medskip
\noindent\textbf{Examples:}
\begin{itemize}[nosep]
  \item \textit{It is \textbf{important} that she \textbf{be} here on time.}
  \item \textit{It is \textbf{vital} that you \textbf{follow} the instructions.}
  \item \textit{It is \textbf{necessary} that he \textbf{attend} the meeting.}
  \item \textit{It is \textbf{imperative} that they \textbf{leave} immediately.}
\end{itemize}

\section*{D. \textit{Wish} and unreal situations}
\begin{itemize}
  \item Use \textbf{wish + past simple} to express regret about the present: 
  \begin{itemize}
    \item \textit{I wish I \textbf{were} rich.}
    \item \textit{She wishes she \textbf{spoke} English better.}
  \end{itemize}
  \item Use \textbf{wish + past perfect} to express regret about the past:
  \begin{itemize}
    \item \textit{I wish I \textbf{had studied} harder.}
  \end{itemize}
\end{itemize}

\section*{E. Expressions: \textit{about time / high time / time}}
These expressions use the past simple to indicate that something should already have happened.
\medskip
\noindent\textbf{Structure:} \textit{(It\textquotesingle s) (about/high) time + subject + past simple.}
\medskip
\noindent\textbf{Examples:}
\begin{itemize}[nosep]
  \item \textit{It\textquotesingle s about time you \textbf{studied}.}
  \item \textit{It\textquotesingle s high time he \textbf{went} to bed.}
  \item \textit{It\textquotesingle s time we \textbf{left}.}
\end{itemize}



\chapter{Negation}

\section*{a. Definition}

Negation in English is the grammatical process used to make a sentence negative. It is formed using negative markers such as \textbf{not}, \textbf{no}, \textbf{never}, or negative pronouns like \textbf{nobody}, \textbf{nothing}, etc. Negation can apply to verbs, nouns, pronouns, adverbs, or objects.

\section*{b. Negation of Nouns}

\begin{tabular}{|m{3cm}|m{7cm}|m{5cm}|}
\hline
\textbf{Word} & \textbf{Definition / Form / Use} & \textbf{Example} \\
\hline
No & Determiner meaning ``not any''; used before singular or plural nouns. & There is \textbf{no} milk left. \\
\hline
Neither & Used with singular nouns meaning ``not one and not the other''. & \textbf{Neither} option is correct. \\
\hline
Not many & Used with countable plural nouns to express a small quantity. & There are \textbf{not many} chairs in the room. \\
\hline
Not much & Used with uncountable nouns to express a small amount. & There is \textbf{not much} time left. \\
\hline
None of - Neither of & If the noun is preceded by the article THE or by a possessive (my, your, his, etc. or jonh's, peter's, etc.) we use the negative quantifiers \textbf{None of} or \textbf{Neither of}. & \textbf{None of} the students passed the test. \newline \textbf{Neither of} my parents can speak English \\
\hline
\end{tabular}


\section*{c. Negative Pronouns}
\begin{tabular}{|m{3cm}|m{7cm}|m{5cm}|}
\hline
\textbf{Word} & \textbf{Definition / Form / Use} & \textbf{Example} \\
\hline
Nobody & Means ``no person''. Subject or object. & \textbf{Nobody} came to the meeting. \\
\hline
No one & Same as ``nobody'', more formal. & \textbf{No one} answered the phone. \\
\hline
Nothing & Means ``no thing''; refers to objects or situations. & There is \textbf{nothing} to do here. \\
\hline
\end{tabular}

\section*{d. Adverb Negations}
\begin{tabular}{|m{3cm}|m{7cm}|m{5cm}|}
\hline
\textbf{Adverb / Structure} & \textbf{Definition / Form / Use} & \textbf{Example} \\
\hline
Never & Means ``at no time''. Can trigger inversion in formal style. & \textbf{Never have I} seen such beauty. \newline John has \textbf{never} seen that picture \\
\hline
Hardly ever & Means ``almost never''; very low frequency. & I \textbf{hardly ever} eat fast food. \\
\hline
Neither ... nor & Connects two negative alternatives; formal. & She likes \textbf{neither} coffee \textbf{nor} tea. \\
\hline
At all & Reinforces a negative verb; used at the end. & I don’t understand \textbf{at all}. \\
\hline
\end{tabular}

\section*{e. Objects of Negation with Verbs}
\begin{tabular}{|m{3cm}|m{7cm}|m{5cm}|}
\hline
\textbf{Word} & \textbf{Definition / Form / Use} & \textbf{Example} \\
\hline
Any & Means ``not even a small amount/number'' in negatives. & She didn’t buy \textbf{any} bread. \\
\hline
Anybody & Means ``any person'' in negative contexts. & I don’t know \textbf{anybody} here. \\
\hline
Anyone & Same as anybody (more formal). & He didn’t invite \textbf{anyone}. \\
\hline
Anything & Means ``any thing'' in negative sentences. & They didn’t say \textbf{anything}. \\
\hline
\end{tabular}







%========================
% TOEIC Exam 
%========================

\newpage 
\begin{center}
    {\Huge \textbf{TOEIC Exam Presentation}}\\[6pt]
    \hrule
\end{center}

\vspace{1em}






\chapter{TOEIC Exam Structure}

The TOEIC Listening and Reading Test is divided into several parts. Each part assesses specific English comprehension and usage skills. 

\section{Part 1 : Photographs [Questions 1-10]}

\textbf{Description:}
\begin{itemize}
    \item In this part of the test, you will see \textbf{photographs} of people, places, objects, or actions.
    \item For each photograph, you will hear \textbf{four statements (A, B, C, D)}.
    \item Only \textbf{one statement correctly describes} what you can see in the photograph.
\end{itemize}

\vspace{0.3cm}

\textbf{Self Questions to Help You:}
\begin{itemize}
    \item Where was the picture taken?
    \item Who are the people in the picture?
    \item What is happening in the picture?
\end{itemize}

\vspace{0.3cm}

\textbf{Tips to Maximize Your Score:}
\begin{itemize}
    \item \textbf{Tip 1:} Listen carefully for \textbf{informal spoken English} contractions: \textit{he’s, she’s, it’s, they’re, there’s, I’ll, it’ll, gonna, I’ve, I’d (I would)}.
    \item \textbf{Tip 2:} Focus on \textbf{details of actions and positions} (e.g. “A man is holding a cup”, “They are sitting at a table”).
    \item \textbf{Tip 3:} Avoid being misled by \textbf{similar-sounding words} (e.g. “desk” vs. “disk”).
\end{itemize}

\begin{figure}[h]
    \centering
    \includegraphics[width=0.8\textwidth]{part1.png}
    \caption{Part 1}
    \label{fig:toeic_chart}
\end{figure}



\section{Part 2 : Question-Response [Questions 11–40]}

\textbf{Description:}
\begin{itemize}
    \item In this part of the test, you will hear a \textbf{question from one speaker}, followed by \textbf{three possible responses} from another speaker.
    \item You must choose the response that \textbf{best fits the question}.
    \item No text will appear on the screen – you must rely entirely on listening comprehension.
\end{itemize}

\vspace{0.3cm}

\textbf{Self Questions to Help You:}
\begin{itemize}
    \item \textbf{Time:} What time / When?
    \item \textbf{Location:} Where is it?
    \item \textbf{Reason:} Why?
    \item \textbf{Identity:} Who?
\end{itemize}

\vspace{0.3cm}

\textbf{Tips to Maximize Your Score:}
\begin{itemize}
    \item \textbf{Tip 1:} Look out for \textbf{WH-question words} such as:
    \textit{who (people), what (object/action), when (time), why (reason), where (location)}.
    \item \textbf{Tip 2:} Focus on \textbf{understanding the reason behind the speaker’s question} — not just the words.
    \item \textbf{Tip 3:} Avoid choosing responses that simply \textbf{repeat words} from the question. They are often incorrect.
    \item \textbf{Tip 4:} Listen for \textbf{intonation and context clues} to identify if the question is about time, place, or reason.
\end{itemize}

\begin{figure}[h]
    \centering
    \includegraphics[width=0.8\textwidth]{part2.png}
    \caption{Part 2}
    \label{fig:Part2}
\end{figure}



\section{Part 3 : Conversations [Questions 41–70]}

\textbf{Description:}
\begin{itemize}
    \item In this part of the test, you will hear short \textbf{conversations between two or three people}.
    \item You will be asked to answer \textbf{three questions} about each conversation.
    \item Each question has \textbf{four answer choices (A, B, C, D)}.
    \item Some questions focus on \textbf{specific details} mentioned in the conversation.
    \item Others require you to \textbf{visualize the setting} and \textbf{identify who the speakers are}.
\end{itemize}

\vspace{0.3cm}

\textbf{Self Questions to Help You:}
\begin{itemize}
    \item Who are the speakers?
    \item Where are the speakers?
    \item What is the conversation about?
    \item What are they doing?
    \item What is their relationship?
\end{itemize}

\vspace{0.3cm}

\textbf{Tips to Maximize Your Score:}
\begin{itemize}
    \item \textbf{Tip 1:} Try to \textbf{read the questions before} listening to the conversation — this helps you know what information to listen for.
    \item \textbf{Tip 2:} Pay attention to \textbf{context clues and vocabulary words} that reveal the setting:
    \begin{itemize}
        \item \textit{computer, e-mail, meeting, desk} $\Rightarrow$ the conversation likely takes place in an \textbf{office}.
    \end{itemize}
    \item \textbf{Tip 3:} Identify the \textbf{purpose of the conversation} (e.g., requesting information, scheduling, giving instructions).
    \item \textbf{Tip 4:} Listen for \textbf{tone and attitude} to understand relationships between speakers (e.g., manager–employee, customer–service agent).
\end{itemize}

\begin{figure}[h]
    \centering
    \includegraphics[width=0.8\textwidth]{part3.png}
    \caption{Part 3}
    \label{fig:Part3}
\end{figure}



\section{Part 4 : Talks [Questions 71–100]}

\textbf{Description:}
\begin{itemize}
    \item In this part of the test, you will hear a series of \textbf{short talks or monologues} given by a single speaker.
    \item Each talk is followed by \textbf{three questions}, and each question has \textbf{four possible answers (A, B, C, D)}.
    \item The talks cover everyday situations such as \textbf{announcements, news reports, public messages, or recorded instructions}.
    \item Some questions focus on \textbf{specific details}, while others test your understanding of the \textbf{overall purpose or context}.
\end{itemize}

\vspace{0.3cm}

\textbf{Self Questions to Help You:}
\begin{itemize}
    \item Who is speaking?
    \item What is the purpose of the talk?
    \item Where is the speaker?
    \item Who is the intended audience?
    \item What is the main idea or topic?
\end{itemize}

\vspace{0.3cm}

\textbf{Tips to Maximize Your Score:}
\begin{itemize}
    \item \textbf{Tip 1:} \textbf{Read the questions before} listening to focus on key information.
    \item \textbf{Tip 2:} Pay attention to \textbf{context clues} such as place names, job titles, or company names to identify the situation.
    \item \textbf{Tip 3:} Listen for \textbf{signal words} that indicate the structure of the talk:
    \begin{itemize}
        \item \textit{First, next, finally} → sequence or procedure.
        \item \textit{Because, due to} → reason or cause.
        \item \textit{So, therefore} → conclusion or result.
    \end{itemize}
    \item \textbf{Tip 4:} Focus on \textbf{the speaker’s purpose}: to inform, to invite, to advertise, or to instruct.
    \item \textbf{Tip 5:} Don’t worry if you miss a word — focus on the \textbf{general meaning} and \textbf{tone}.
\end{itemize}

\begin{figure}[h]
    \centering
    \includegraphics[width=0.8\textwidth]{part4.png}
    \caption{Part 4}
    \label{fig:part4}
\end{figure}



\section{Part 5 : Incomplete Sentences [Questions 101–130]}

\textbf{Description:}
\begin{itemize}
    \item In this part of the test, you will read \textbf{incomplete sentences} with a missing word or phrase.
    \item Each question has \textbf{four answer choices (A, B, C, D)}.
    \item You must choose the word or phrase that \textbf{correctly completes the sentence} according to grammar and meaning.
    \item This section tests your knowledge of \textbf{grammar}, \textbf{vocabulary}, and \textbf{collocations (word combinations)}.
\end{itemize}

\vspace{0.3cm}

\textbf{Self Questions to Help You:}
\begin{itemize}
    \item What type of word is missing? (noun, verb, adjective, preposition…)
    \item What tense or form fits grammatically?
    \item Does the sentence test vocabulary or grammar?
    \item Are there key words that indicate a specific grammatical rule?
\end{itemize}

\vspace{0.3cm}

\textbf{Tips to Maximize Your Score:}
\begin{itemize}
    \item \textbf{Tip 1:} Read the entire sentence before looking at the options — understand the context first.
    \item \textbf{Tip 2:} Identify \textbf{signal words} (e.g. “since”, “while”, “although”) that help determine tense or structure.
    \item \textbf{Tip 3:} Be aware of common grammar traps:
    \begin{itemize}
        \item Verb tense consistency (e.g. has been / was / will be)
        \item Prepositions (e.g. depend \textbf{on}, responsible \textbf{for})
        \item Articles and quantifiers (e.g. a / an / the / some / any)
    \end{itemize}
    \item \textbf{Tip 4:} Learn common \textbf{TOEIC collocations}:  
    \textit{meet a deadline, take responsibility, make a decision, provide information}.
    \item \textbf{Tip 5:} Don’t overthink — if two options sound correct, choose the one that fits both \textbf{grammar and logic}.
\end{itemize}

\begin{figure}[h]
    \centering
    \includegraphics[width=0.8\textwidth]{part5.png}
    \caption{Part 5}
    \label{fig:part5}
\end{figure}



\section{Part 6 : Text Completion [Questions 131–146]}

\textbf{Description:}
\begin{itemize}
    \item In this part of the test, you will read \textbf{short texts} such as e-mails, letters, or announcements with several blanks.
    \item Each text contains \textbf{three or four blanks}, and for each blank, you must choose the best answer from \textbf{four options (A, B, C, D)}.
    \item The correct answer depends on both \textbf{grammar} and \textbf{context}.
    \item This section tests your ability to understand the \textbf{flow of ideas} and \textbf{logical connections} between sentences.
\end{itemize}

\vspace{0.3cm}

\textbf{Self Questions to Help You:}
\begin{itemize}
    \item What type of text is this? (email, memo, notice, article…)
    \item What is the main purpose of the text?
    \item Does the blank need a word for grammar, vocabulary, or logic?
    \item What connects the sentence before and after the blank?
\end{itemize}

\vspace{0.3cm}

\textbf{Tips to Maximize Your Score:}
\begin{itemize}
    \item \textbf{Tip 1:} Read the entire text once to understand the general meaning \textbf{before filling in the blanks}.
    \item \textbf{Tip 2:} Identify \textbf{transition words} and \textbf{connectors}:  
    \textit{however, therefore, in addition, as a result, although}.
    \item \textbf{Tip 3:} Pay attention to \textbf{pronouns and references} (e.g., “it”, “they”, “this”) that link sentences together.
    \item \textbf{Tip 4:} Be careful with \textbf{verb tenses} — they must remain consistent within the same text.
    \item \textbf{Tip 5:} For business-style texts, learn useful expressions such as:  
    \textit{We are pleased to inform you…, Please be advised that…, Thank you for your cooperation.}
    \item \textbf{Tip 6:} Eliminate options that change the meaning or tone of the message.
\end{itemize}

\begin{figure}[h]
    \centering
    \includegraphics[width=0.8\textwidth]{part6.png}
    \caption{Part 6}
    \label{fig:part6}
\end{figure}



\section{Part 7 : Reading Comprehension [Questions 147–200]}

\textbf{Description:}
\begin{itemize}
    \item In this part of the test, you will read a variety of texts such as \textbf{articles, advertisements, e-mails, notices, forms, and reports}.
    \item Each set of questions is based on a \textbf{single passage} or a set of \textbf{two or three related passages}.
    \item You will answer \textbf{multiple-choice questions} about the content, purpose, and details of the texts.
    \item This section tests your \textbf{reading speed}, \textbf{comprehension}, and \textbf{ability to infer information}.
\end{itemize}

\vspace{0.3cm}

\textbf{Self Questions to Help You:}
\begin{itemize}
    \item What type of document is this? (email, announcement, article, schedule, form…)
    \item Who wrote it and for whom?
    \item What is the main purpose or topic?
    \item What details or numbers are mentioned?
    \item What can be inferred or implied (not directly said)?
\end{itemize}

\vspace{0.3cm}

\textbf{Tips to Maximize Your Score:}
\begin{itemize}
    \item \textbf{Tip 1:} Skim the passage quickly to get the \textbf{main idea} before reading the questions.
    \item \textbf{Tip 2:} Read the \textbf{questions first} — this helps you know what to look for in the text.
    \item \textbf{Tip 3:} Use \textbf{keywords} from the question to locate information quickly.
    \item \textbf{Tip 4:} Watch out for \textbf{paraphrasing}: the same idea may be expressed with different words.
    \item \textbf{Tip 5:} Focus on the \textbf{logic of the passage}: cause → effect, problem → solution, request → response.
    \item \textbf{Tip 6:} For double or triple passages:
    \begin{itemize}
        \item Identify the \textbf{relationship} between texts (e.g., email → reply, article → comment).
        \item Answer questions that require \textbf{cross-referencing information}.
    \end{itemize}
    \item \textbf{Tip 7:} Manage your time: around \textbf{75 minutes for 54 questions}, so don’t get stuck — move on if unsure.
\end{itemize}

\begin{figure}[h]
    \centering
    \includegraphics[width=0.8\textwidth]{part7.png}
    \caption{Part 7}
    \label{fig:part7}
\end{figure}





\chapter{Tips for TOEIC}


\section{Common Erros in Part 5}


\renewcommand{\arraystretch}{1.4}
\begin{longtable}{|p{6cm}|p{3cm}|p{6cm}|}
\hline
\textbf{Question} & \textbf{Best Answer} & \textbf{Explanation} \\
\hline
\endfirsthead

\hline
\textbf{Question} & \textbf{Best Answer} & \textbf{Explanation} \\
\hline
\endhead

Aurora Furnishings is finding it difficult to make a profit in its \_\_\_\_ competitive market. 
\newline \textbf{choices :} increases , increased , increasingly , increase 
& increasingly 
& We use the pattern: \textbf{adverb + adjective + noun}. \\
\hline

The maintenance team’s repair requests should be \_\_\_\_ in groups according to the urgency.
\newline \textbf{choices :} organizing , organize , organized , organizes 
& organized 
& After \textbf{modal + be} (here “should be”), we use the \textbf{past participle} for passive structures. \\
\hline

Due to the high volume of foot traffic, the shop must polish its floors more \_\_\_\_ than usual during the peak season.
\newline \textbf{choices :} frequent , frequented , frequency , frequently 
& frequently 
& We use an \textbf{adverb} to modify a verb, following the pattern: \textbf{verb + adverb}. \\
\hline

\_\_\_\_ the building has an excellent location and a modern interior, it is popular among visitors.
\newline \textbf{choices : } In view of , Provided that , Other than , Seeing that
& seeing that
& \textbf{“Seeing that”} introduces a reason clause meaning \textbf{“since / because”}. \\
\hline

The landlord raised the monthly rent for the first time in several years, and \_\_\_\_ so.
\newline \textbf{choices :} reasonable , reasonably , reason , reasons 
& reasonably 
& We use the pattern: \textbf{and + adverb + so}. \\
\hline

The green light on the side of the water purifier lights up \_\_\_\_ the filter needs to be replaced.
\newline \textbf{choices :} likewise , whenever , therefore , whereas
& whenever 
& We use \textbf{“whenever”} to mean \textbf{“every time that”}. \\
\hline

The clerk said that \_\_\_\_ fifteen customers had been waiting outside the store for it to open.
\newline \textbf{choices :} rougher , roughly , rough , roughness
& roughly 
& We use the pattern : \textbf{adverb + number + noun}. \\
\hline

The company plans on \_\_\_\_ the salespeople for the expenses they incurred while attending the conference.
\newline \textbf{choices :} reimbursement , reimbursed , reimburse , reimbursing
& reimbursing
& We use the pattern : \textbf{plan on + V-ing}. \\
\hline

The heavy rains \_\_\_\_ problems for travelers if not for the intervention of the traffic police.
\newline \textbf{choices :} will cause , would have caused , will have caused , can cause 
& would have caused 
& Pattern : \textbf{if not for + noun → Third conditional : would have + PP}. \\
\hline

Stir the flour into the batter as \_\_\_\_ as possible to prevent the dough from being lumpy.
\newline \textbf{choices :} quickest , quick , quicker , quickly
& quickly
& Pattern : \textbf{as + adverb + as possible}. \\
\hline


The lecturer \_\_\_\_\_ on the country's struggle for independence when he gives his talk.
\newline \textbf{choices :} has focused , is focused , will focus , focusing 
& will focus 
& Pattern : \textbf{Futur Simple +  when Prsent Simple}. \\
\hline

The manufacturing plant that was damaged in the typhoon should \_\_\_\_ its operations later this month.
\newline \textbf{choices} : resumed , be resuming , had resumed , resuming 
& be resuming 
& After \textbf{should} there is always a base verb \\
\hline 


\end{longtable}



\begin{table}[h]
\centering
\resetrownum
\begin{tabular}{@{}p{0.5cm} p{4cm} p{6.5cm} p{6.5cm}@{}}
\toprule
\text{Num} & \textbf{Word / Structure} & \textbf{Definition } & \textbf{Example} \\
\midrule
\rownumber & \textit{under which} & on utilise \textbf{under which} pour parler d'une règle ou loi & 
The policy \textbf{under which} refunds are processed has changed. \\
\midrule
\rownumber & \textit{prefer + V-ing + to + V-ing} & Compare two activities / preferences & 
I prefer \textbf{reading} reports to \textbf{attending} long meetings. \\
\midrule
\rownumber & \textit{has/have been + past participle} & Present perfect passive: past action with a present result & 
The contract \textbf{has been signed}. \\
\midrule
\rownumber & \textit{delighted with} & Pleased/very happy about something &
The client was \textbf{delighted with} the results. \\
\midrule
\rownumber & \textit{jewellery} (uncountable) & Uncountable noun; no plural *jewelleries*; verb takes singular &
Her \textbf{jewellery is} expensive. \\
\midrule
\rownumber & \textit{once} + present simple & Use present simple after “once” even for future reference &
\textbf{Once he arrives}, we will start. \\
\midrule
\rownumber & \textit{could} (possibility) & Express possibility/conditional suggestion &
If we had more time, we \textbf{could} expand the study. \\
\midrule
\rownumber & \textit{object to} + N / V-ing & “Object” always with \textit{to}; if a verb follows, use -ing &
They objected \textbf{to} the proposal / \textbf{to working} late. \\
\midrule
\rownumber & \textit{be keen on} + N/V-ing; \textit{keen to} + V & Enthusiasm/interest (\textit{keen on}); willingness/intent (\textit{keen to}) &
She is \textbf{keen on learning} Python; she is \textbf{keen to learn} Python. \\
\bottomrule
\end{tabular}
\end{table}




%========================
% Vocabulaire  
%========================

\newpage 
\begin{center}
    {\Huge \textbf{Vocabulaire}}\\[6pt]
    \hrule
\end{center}

\vspace{1em}


%========================
\chapter{Vocabulaire \& collocations}
%========================




\begin{itemize}
  \item Personnel 
  \item Offices 
  \item Purchasing 
  \item Dinning Out 
  \item General Businnes 
  \item Entertainment 
  \item Manufacturing
  \item Travel 
\end{itemize}


\begin{center}
\begin{tabular}{@{}L{4.5cm} L{10cm}@{}}
\toprule
\textbf{Expression} & \textbf{Exemple TOEIC} \\
\midrule
meet a deadline & We must \textbf{meet the deadline} by Friday. \\
make a decision & Management will \textbf{make a decision} tomorrow. \\
apply for a position & She \textbf{applied for} the marketing position. \\
be responsible for & He is \textbf{responsible for} quality control. \\
in accordance with & The policy is \textbf{in accordance with} regulations. \\
\bottomrule
\end{tabular}
\end{center}

%========================
\section{Contractions \& connecteurs}
%========================

\subsection*{Contractions fréquentes (écoute TOEIC)}
\begin{center}
\begin{tabular}{@{}L{5cm} L{9.5cm}@{}}
\toprule
\textbf{Forme} & \textbf{Exemple} \\
\midrule
I'm, you're, he's, we're, they're & \textit{They're ready for the call.} \\
don't, doesn't, didn't & \textit{She doesn't agree.} \\
I'll, you'll, we'll, they'll & \textit{I'll send the report.} \\
can't, won't, shouldn't, couldn't & \textit{We can't attend today.} \\
\bottomrule
\end{tabular}
\end{center}

\subsection*{Connecteurs logiques (écrit/oral)}
\begin{itemize}
  \item \textbf{Therefore, consequently}: conséquence \;—\; \textit{It was delayed; therefore, we rescheduled.}
  \item \textbf{However, nevertheless}: opposition \;—\; \textit{Expensive; however, effective.}
  \item \textbf{Moreover, in addition}: addition \;—\; \textit{Moreover, we reduced costs.}
\end{itemize}



\section*{Vocabulary  - Dining out}


\begin{center}
\renewcommand{\arraystretch}{1.4}
\begin{tabular}{|p{3cm}|p{3cm}|p{8cm}|}
\hline
\textbf{Word / Expression} & \textbf{Traduction (FR)} & \textbf{Example Sentence} \\
\hline
parlor & salon / salle de réception & We had ice cream in a small parlor downtown. \\
\hline
staff & personnel / employés & The hotel staff were very friendly. \\
\hline
shape & forme / aspect & This table has a round shape. \\
\hline
the dough & la pâte & Knead the dough until it becomes soft. \\
\hline
in on the secret & au courant du secret & Only a few people are in on the secret. \\
\hline
apple pie & tarte aux pommes & I baked an apple pie for dessert. \\
\hline
cater & fournir / approvisionner en nourriture & They cater food for weddings and big events. \\
\hline
banquet & banquet / réception & The king invited us to a royal banquet. \\
\hline
tastes & goûts / préférences & Everyone has different tastes in music. \\
\hline
dietary & alimentaire / diététique & She follows strict dietary rules. \\
\hline
county & comté / département & He lives in Orange County, California. \\
\hline
a catering service & service de traiteur & We hired a catering service for the company party. \\
\hline
leisure & loisirs / temps libre & I read books in my leisure time. \\
\hline
excerpts & extraits / passages & The teacher read excerpts from the novel. \\
\hline
fall off & tomber de / décrocher / diminuer & The picture fell off the wall. \\
\hline
curtains & rideaux & She closed the curtains before going to bed. \\
\hline
ring & sonner / appeler & The phone rang early this morning. \\
\hline
fell asleep & s’est endormi & He fell asleep during the meeting. \\
\hline
\end{tabular}
\end{center}

% \begin{center}
% \renewcommand{\arraystretch}{1.4}
% \begin{tabular}{|p{3cm}|p{3cm}|p{8cm}|}
% \hline
% \textbf{Word / Expression} & \textbf{Arabic Translation} & \textbf{Example Sentence} \\
% \hline
% parlor & صالون / غرفة استقبال & We had ice cream in a small parlor downtown. \\
% \hline
% staff & الموظفون / العاملون & The hotel staff were very friendly. \\
% \hline
% shape & شكل / هيئة & This table has a round shape. \\
% \hline
% the dough & العجين & Knead the dough until it becomes soft. \\
% \hline
% in on the secret & مطلع على السر & Only a few people are in on the secret. \\
% \hline
% apple pie & فطيرة التفاح & I baked an apple pie for dessert. \\
% \hline
% cater & يقدم الطعام / يخدم & They cater food for weddings and big events. \\
% \hline
% banquet & مأدبة / وليمة & The king invited us to a royal banquet. \\
% \hline
% tastes & المذاقات / الأذواق & Everyone has different tastes in music. \\
% \hline
% dietary & غذائي / خاص بالنظام الغذائي & She follows strict dietary rules. \\
% \hline
% county & مقاطعة & He lives in Orange County, California. \\
% \hline
% a catering service & خدمة تقديم الطعام & We hired a catering service for the company party. \\
% \hline
% leisure & وقت الفراغ / الترفيه & I read books in my leisure time. \\
% \hline
% excerpts & مقتطفات / فقرات مختارة & The teacher read excerpts from the novel. \\
% \hline
% fall off & يسقط من / ينخفض & The picture fell off the wall. \\
% \hline
% curtains & الستائر & She closed the curtains before going to bed. \\
% \hline
% ring & يرنّ / يتصل & The phone rang early this morning. \\
% \hline
% fell asleep & غفا / نام & He fell asleep during the meeting. \\
% \hline
% \end{tabular}
% \end{center}




\vfill
\begin{center}
\small \textit{Dernière mise à jour : \today \;—\; Ce document est un mémo de révision TOEIC.}
\end{center}

\end{document}
